\let\negmedspace\undefined
\let\negthickspace\undefined
\documentclass[journal]{IEEEtran}
\usepackage[a5paper, margin=10mm, onecolumn]{geometry}
%\usepackage{lmodern} % Ensure lmodern is loaded for pdflatex
\usepackage{tfrupee} % Include tfrupee package

\setlength{\headheight}{1cm} % Set the height of the header box
\setlength{\headsep}{0mm}     % Set the distance between the header box and the top of the text
\usepackage{xparse}
\usepackage{gvv-book}
\usepackage{gvv}
\usepackage{cite}
\usepackage{amsmath,amssymb,amsfonts,amsthm}
\usepackage{algorithmic}
\usepackage{graphicx}
\usepackage{textcomp}
\usepackage{xcolor}
\usepackage{txfonts}
\usepackage{listings}
\usepackage{enumitem}
\usepackage{mathtools}
\usepackage{gensymb}
\usepackage{comment}
\usepackage[breaklinks=true]{hyperref}
\usepackage{tkz-euclide} 
\usepackage{listings}
% \usepackage{gvv}                                        
\def\inputGnumericTable{} 
\usepackage[latin1]{inputenc}                                
\usepackage{color}                                            
\usepackage{array}                                            
\usepackage{longtable}                                       
\usepackage{calc}                                             
\usepackage{multirow}                                         
\usepackage{hhline}                                           
\usepackage{ifthen}                                           
\usepackage{lscape}

\begin{document}

\bibliographystyle{IEEEtran}
\vspace{3cm}

\title{2007-XE-35-51}
\author{EE24BTECH11006 - Arnav Mahishi}
% \maketitle
% \newpage
% \bigskip
{\let\newpage\relax\maketitle}
\begin{enumerate}
\item{
The minimum number of terms required in the series expansion of $e^x$ to evaluate at $x=1$ correct up to $3$ places of decimals is
\begin{multicols}{4}
\begin{enumerate}
\item$8$
\item$7$
\item$6$
\item$5$
\end{enumerate}
\end{multicols}
}
\item{
The iteration scheme $x_{n+1}=\frac{1}{1+x_n^2}$ converges to a real number $x$ in the interval $\brak{0,1}$ with $x_o=0.5$. The value of $x$ correct up to $2$ places of decimal is equal to
\begin{multicols}{4}
\begin{enumerate}
\item $0.65$
\item $0.68$
\item $0.73$
\item $0.80$
\end{enumerate}
\end{multicols}}
\item{
If the diagonal elements of a lower triangular square matrix $A$ are all different from zero, then the matrix $A$  will always be 
\begin{multicols}{4}
\begin{enumerate}
\item symmetric
\item non-symmetric
\item singular
\item non-singular
\end{enumerate}
\end{multicols}
}
\item{
If the two eigen values of the matrix $M=\myvec{2&6&0\\1&p&0\\0&0&3}$ are $-1$ and $4$, then the value of $p$ is 
\begin{multicols}{4}
\begin{enumerate}
\item $4$
\item $2$
\item $1$
\item $-1$
\end{enumerate}
\end{multicols}
}
\item{
Consider the system of linear simultaneous equations
\begin{align*}
x+10y &= 5, \\
y+5z &= 1, \\
10x-y+z &= 0
\end{align*}
On applying Gauss-Seidel method the value of $x$ correct up to $4$ decimal places is
\begin{multicols}{4}
\begin{enumerate}
\item $0.0385$
\item $0.0395$
\item $0.0405$
\item $0.0410$
\end{enumerate}
\end{multicols}
}
\item{
The graph of a function $y=f\brak{x}$ passes through the points $\brak{0,-3},\brak{1,-1}$, and $\brak{2,3}$. Using Lagrange interpolation, the value of $x$ at which the curve crosses the $x$-axis is obtained as
\begin{multicols}{4}
\begin{enumerate}
\item $1.375$
\item $1.475$
\item $1.575$
\item $1.675$
\end{enumerate}
\end{multicols}
}
\item{
The equation of the straight line of best fit using the following data 
\begin{table}[h!]   
  \centering
  \begin{tabular}[10pt]{ |c| c| c| c| c| c|}
    \hline
     $x$&$1$&$2$&$3$&$4$&$5$\\ 
    \hline
    $y$&$14$&$13$&$9$&$5$&$2$\\
    \hline 
    \end{tabular}

\end{table}
by the principle of least square is
\begin{multicols}{4}
\begin{enumerate}
\item $y=18-3x$
\item $y=18.1-3.1x$
\item $18.2-3.2x$
\item $18.3-3.3x$
\end{enumerate}
\end{multicols}
}
\item{
On solving the initial value problem $\frac{dy}{dx}=xy^2,y\brak{1}=1$ by Euler's method, the value of $y$ at $x=1.2$ with $h=0.1$ is
\begin{multicols}{4}
\begin{enumerate}
\item $1.1000$
\item $1.1232$
\item $1.2210$
\item $1.2331$
\end{enumerate}
\end{multicols}
}
\item{
The local error of the following schme $y_{n+1}=y_n+\frac{h}{12}\brak{5y'_{n+1} + 8y'_n - y'_{n-1}}$ by comparing with the Taylor series $y_{n+1} = y_n + hy'_n + \frac{h^2}{2!}y''_n + \cdots$ is 
\begin{multicols}{4}
\begin{enumerate}
\item $O\brak{h^4}$
\item $O\brak{h^5}$
\item $O\brak{h^2}$
\item $O\brak{h^3}$
\end{enumerate}
\end{multicols}
}
\item{
The area bounded by the curve $y=1-x^2$ and the $x$-axis from $x=-1$ to $x=1$ using the Trapezoidal rule with step length $h=0.5$ is
\begin{multicols}{4}
\begin{enumerate}
\item $1.20$
\item $1.23$
\item $1.25$
\item $1.33$
\end{enumerate}
\end{multicols}
}
\item{
The iteration scheme $x_{n+1}=\sqrt{a}\brak{1+\frac{3a^2}{x_n^2}}-\frac{3a^2}{x_n}$, $a>0$ converges to the real number
\begin{multicols}{4}
\begin{enumerate}
\item $\sqrt{a}$
\item $a$
\item $a\sqrt{a}$
\item $a^2$
\end{enumerate}
\end{multicols}
}
\item{
If the binary representation of two numbers $m$ and $n$ are $01001101$ and $00101011$, respectively, then the binary representation of $m-n$ is 
\begin{multicols}{4}
\begin{enumerate}
\item $00010010$
\item $00100010$
\item $00111101$
\item $00100001$
\end{enumerate}
\end{multicols}
}
\item{
Which of the follwing statements are true in a C program?
\text{P: A local variable is used only within the block where it is defined, and its sub-blocks}\\
\text{Q: Global variables are declared outside the scope of all blocks}\\
\text{R: Extern variables are used by linkers for sharing between other compilation units}\\
\text{S: By default, all global variables are extern variables}
\begin{multicols}{4}
\begin{enumerate}
\item P and Q
\item P,Q and R
\item P,Q and S
\item P,Q,R and S
\end{enumerate}
\end{multicols}
}
\item{
Consider the following recursive function $g\brak{}$\\
Recursive integer function $g\brak{m, n}$ result $\brak{r}$\\
    \textit{integer :: m, n\\
    if $\brak{n == 0}$ then\\
        r = m\\
    else if $\brak{m \leq 0}$ then\\
        r = n + 1\\
    else if $\brak{\brak{n - n / 2 * 2} == 1}$ then\\
        r = g\brak{m - 1, n + 1}\\
    else\\
        r = g\brak{m - 2, n / 2}\\
    end if\\
end\\}
Which value will be returned if the function $g$ is called with $6,6$?
\begin{enumerate}
\item 2 
\item 4
\item 6
\item 8
\end{enumerate}
}
\item{
If the following function is called with $x=1$\\
\textit{
real function print\textunderscore value$\brak{x}$\\
    real :: x, sum, term\\
    integer :: i\\
    i = $0$\\
    sum = $2.0$\\
    term = $1.0$\\
    do while (term $> 0.00001$)\\
        term = $x$ * term / (i + 1)\\
        sum = sum + term\\
        i = i + 1\\
    end do\\
    print\textunderscore value = sum\\
end\\
}
The value returned will be close to
\begin{multicols}{4}
\begin{enumerate}
\item $log_e2$
\item $log_e3$
\item $1+e$
\item $e$
\end{enumerate}
\end{multicols}
}
\item{
Consider the following C program\\
\textit{
include $<$stdio.h$>$\\
include $<$string.h$>$\\
void main()\\
{\\
    char s$\sbrak{80}$, *p;\\
    int sum = 0;\\
    p = s;\\
    gets$\brak{s}$;\\
    while $\brak{*p}$\\
    {\\
        if $\brak{*p == \text{'1'}}$\\
            sum = 2 * sum + 1;\\
        else if $\brak{*p == \text{'0'}}$\\
            sum = sum * 2;\\
        else\\
            printf$\brak{\text{"invalid string"}}$;\\
        p++;\\
    }\\
    printf("$\%$d", sum);\\
}\\
Which number will be printed if the input string is $10110$
\begin{multicols}{4}
\begin{enumerate}
    \item $31$
    \item $28$
    \item $25$
    \item $22$
\end{enumerate}
\end{multicols}
}
\item{
Consider the following C program segment\\
\textit{
\#include $<$stdio.h$>$\\
void print\_mat$\brak{\text{int mat$\sbrak{1}\sbrak{3}$}}$\\\\
void main$\brak{}$\{\\
int i,j,sum=0;\\
int m$\sbrak{3}\sbrak{3}$=\{\{1, 3, 5\}, \{7, 9, 11\}, \{13, 15, 17\}\};\\
for$\brak{i=0;i<3;i++}$\{\\
for$\brak{j=2;j>1;j--}$\{\\
sum+=m$\sbrak{i}\sbrak{j}*m\sbrak{i}\sbrak{j-1}$;\\
printf$\brak{\text{"\%d", sum}}$\\
print\_mat$\brak{m}$;//FUNCTION CALL\}\\\\
void print\_mat$\brak{\text{int mat$\sbrak{}\sbrak{3}$}}$$\{\\
int *p$\sbrak{3}$=&mat$\sbrak{l}$;\\
printf$\brak{\text{"\%d and \%d",*p$\sbrak{1}$,*p$\sbrak{2}$}}$;\\\}

}
The value of sum that will be printed by the above program is\\
\begin{multicols}{4}
\begin{enumerate}
    \item $369$
    \item $361$
    \item $303$
    \item $261$
\end{enumerate}
\end{multicols}
}
\end{enumerate}
\end{document}e
