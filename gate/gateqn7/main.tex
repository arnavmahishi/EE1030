\let\negmedspace\undefined
\let\negthickspace\undefined
\documentclass[journal]{IEEEtran}
\usepackage[a5paper, margin=10mm, onecolumn]{geometry}
%\usepackage{lmodern} % Ensure lmodern is loaded for pdflatex
\usepackage{tfrupee} % Include tfrupee package

\setlength{\headheight}{1cm} % Set the height of the header box
\setlength{\headsep}{0mm}     % Set the distance between the header box and the top of the text
\usepackage{xparse}
\usepackage{gvv-book}
\usepackage{gvv}
\usepackage{cite}
\usepackage{amsmath,amssymb,amsfonts,amsthm}
\usepackage{algorithmic}
\usepackage{graphicx}
\usepackage{textcomp}
\usepackage{xcolor}
\usepackage{txfonts}
\usepackage{listings}
\usepackage{enumitem}
\usepackage{mathtools}
\usepackage{gensymb}
\usepackage{comment}
\usepackage[breaklinks=true]{hyperref}
\usepackage{tkz-euclide} 
\usepackage{listings}
% \usepackage{gvv}                                        
\def\inputGnumericTable{} 
\usepackage[latin1]{inputenc}                                
\usepackage{color}                                            
\usepackage{array}                                            
\usepackage{longtable}                                       
\usepackage{calc}                                             
\usepackage{multirow}                                         
\usepackage{hhline}                                           
\usepackage{ifthen}                                           
\usepackage{lscape}

\begin{document}

\bibliographystyle{IEEEtran}
\vspace{3cm}

\title{2012-EE-1-13}
\author{EE24BTECH11006 - Arnav Mahishi}
% \maketitle
% \newpage
% \bigskip
{\let\newpage\relax\maketitle}
\begin{enumerate}
\item{
Two independent random variables $X$ and $Y$ are uniformly distributed in the interval $\sbrak{-1,1}$. The probability that max$\sbrak{X,Y}$ is less than $\frac{1}{2}$ is 
\begin{multicols}{4}
\begin{enumerate}
\item $\frac{3}{4}$
\item $\frac{9}{16}$
\item $\frac{1}{4}$
\item $\frac{2}{3}$
\end{enumerate}
\end{multicols}
}
\item{
If $x=\sqrt{-1}$, then the value of $x^x$ is
\begin{multicols}{4}
\begin{enumerate}
\item $e^{\frac{-\pi}{2}}$ 
\item $e^{\frac{\pi}{2}}$ 
\item $x$
\item $1$ 
\end{enumerate}
\end{multicols}}
\item{
Given $f\brak{z}=\frac{1}{z+1}-\frac{2}{z+3}$. If $C$ is a counterclockwise path in the $z-plane$ such that $\abs{z+1}=1$, the value of $\frac{1}{2\pi j}\oint_cf\brak{z}dz$ is
\begin{multicols}{4}
\begin{enumerate}
\item $-2$
\item $-1$
\item $1$
\item $2$
\end{enumerate}
\end{multicols}
}
\item{
In the circuit shown below, the current through the inductor is
\begin{figure}[H]
\centering
\resizebox{3cm}{!}{%
\begin{circuitikz}
\tikzstyle{every node}=[font=\small]
\draw (6,8.75) to[sinusoidal voltage source, sources/symbol/rotate=auto] (8,8.75);
\draw (6,8.75) to[R] (7.75,10.5);
\draw (6,8.75) to[C] (7.75,7);
\draw (7.75,7) to[R] (9.5,8.75);
\draw (7.75,7) to[american current source] (7.75,8.75);
\draw (7.75,8.75) to[american current source] (7.75,10.5);
\draw (8,8.75) to[sinusoidal voltage source, sources/symbol/rotate=auto] (9.5,8.75);
\draw (7.75,10.5) to[L ] (9.5,8.75);
\node [font=\normalsize] at (6.25,9.75) {1$\Omega$};
\node [font=\normalsize] at (9.25,10) {j$\Omega$};
\node [font=\normalsize] at (6.25,7.5) {-j$\Omega$};
\node [font=\normalsize] at (9,7.5) {1$\Omega$};
\node [font=\scriptsize] at (7.25,9.25) {$1\angle 0V$};
\node [font=\scriptsize] at (8.35,9.25) {$1\angle 0V$};
\node [font=\small] at (8,8.5) {$1\angle 0A$};
\node [font=\small] at (7.75,10.75) {$1\angle 0A$};
\end{circuitikz}
}%
\label{fig:my_label}
\end{figure}
\begin{multicols}{4}
\begin{enumerate}
\item $\frac{2}{1+j}A$
\item $\frac{-1}{1+j}A$
\item $\frac{1}{1+j}$
\item $0A$
\end{enumerate}
\end{multicols}
}
\item{
The impedence looking into nodes $1$ and $2$ in the given circuit is
\begin{figure}[H]
\centering
\resizebox{5cm}{!}{%
\begin{circuitikz}
\tikzstyle{every node}=[font=\normalsize]
\draw (6.5,7.25) to[short] (6.5,8.5);
\draw (6.5,8.5) to[R] (6.5,9.75);
\draw (6.5,9.75) to[short] (6.5,10.75);
\draw (6.5,10.75) to[short] (8.25,10.75);
\draw (8.25,10.75) to[R] (8.25,9.5);
\draw (8.25,9.5) to[short] (10,9.5);
\draw (6.5,7.25) to[short] (11.5,7.25);
\draw (9.25,9.5) to[R] (9.25,7.25);
\draw (10,9.5) to[short, -o] (10,8.75) ;
\draw (10,7.25) to[short, -o] (10,7.75) ;
\draw (10,11) to[american controlled current source] (10,9.5);
\draw (10,11) to[short] (11.5,11);
\draw (11.5,11) to[short] (11.5,7.25);
\draw [->, >=Stealth] (6.75,10.75) -- (7.75,10.75);
\node [font=\normalsize] at (7.5,11) {$i_b$};
\node [font=\normalsize] at (6,9) {$9k\Omega$};
\node [font=\normalsize] at (8.5,8.5) {$100\Omega$};
\node [font=\normalsize] at (10.25,8.75) {$1$};
\node [font=\normalsize] at (10.25,7.75) {$2$};
\node [font=\normalsize] at (10.8,10.25) {$99i_b$};
\node [font=\normalsize] at (7.75,10) {$1k\Omega$};
\end{circuitikz}
}%

\label{fig:my_label}
\end{figure}
\begin{multicols}{4}
\begin{enumerate}
\item $50\Omega$
\item $100\Omega$
\item $5k\Omega$
\item $10.1k\Omega$
\end{enumerate}
\end{multicols}
}
\item{
A system with transfer function $G\brak{s}=\frac{\brak{s^2+9}\brak{s+2}}{\brak{s+1}\brak{s+3}\brak{s+4}}$ is excited by $\sin\omega t$. The steady-state output of the system is zero at
\begin{multicols}{4}
\begin{enumerate}
\item $\omega=1\frac{rad}{s}$
\item $\omega=2\frac{rad}{s}$
\item $\omega=3\frac{rad}{s}$
\item $\omega=4\frac{rad}{s}$
\end{enumerate}
\end{multicols}
}
\item{
In the sum of products function $f\brak{X,Y,Z}=\sum\brak{2,3,4,5}$, the prime implicants are 
\begin{multicols}{4}
\begin{enumerate}
\item $\overline{X}Y,X\overline{Y}$
\item $\overline{X}Y,X\overline{Y}\overline{Z},X\overline{Y}Z$
\item $\overline{X}Y\overline{Z},\overline{X}YZ,X\overline{Y}$
\item $\overline{X}Y\overline{Z},\overline{X}YZ,X\overline{Y}Z$
\end{enumerate}
\end{multicols}
}
\item{
If $x\sbrak{n}=\brak{\frac{1}{3}}^{\abs{n}}-\brak{\frac{1}{2}}^nu\sbrak{n}$, the region of convergence $\brak{\text{ROC}}$ of its $Z$-transform in the $Z$-plane will be
\begin{multicols}{4}
\begin{enumerate}
\item $\frac{1}{3}\textless\abs{z}\textless 3$
\item $\frac{1}{3}\textless\abs{z}\textless\frac{1}{2}$
\item $\frac{1}{2}\textless\abs{z}\textless 3$
\item $\frac{1}{3}\textless\abs{z}\textless 1$
\end{enumerate}
\end{multicols}
}
\item{
The bus admittance matrix of a three-bus three-line system is\\ $Y=j\myvec{-13&10&5\\10&-18&10\\5&10&-13}$\\If each transmission line between the two buses is represented by an equivalent $\pi$-network, the magnitude of shunt susceptance of the line connecting bus $1$ and $2$ is
\begin{multicols}{4}
\begin{enumerate}
\item $4$
\item $2$
\item $1$
\item $0$
\end{enumerate}
\end{multicols}
}
\item{
The slip of an induction motor normally does not depend on 
\begin{multicols}{2}
\begin{enumerate}
\item rotor speed
\item synchronous speed
\item shaft torque
\item core-less component
\end{enumerate}
\end{multicols}
}
\item{
A two-phase load draws the following phase currents: $i_1\brak{t}=I_m\sin\brak{\omega t-\phi_1}$, $i_2\brak{t}=I_m\sin\brak{\omega t-\phi_2}$. These currents are balanced if $\phi_1$ is equal to
\begin{multicols}{4}
\begin{enumerate}
\item $-\phi_2$
\item $\phi_2$
\item $\brak{\frac{\pi}{2}-\phi_2}$
\item $\brak{\frac{\pi}{2}+\phi_2}$
\end{enumerate}
\end{multicols}
}
\item{
A periodic voltage waveform observed on an oscilloscope across a load is shown. A permanent magnet moving coil\brak{PMMC} meter connected across the same load reads
\begin{figure}[H]
\centering
\resizebox{5cm}{!}{%
\begin{circuitikz}
\tikzstyle{every node}=[font=\normalsize]
\draw [->, >=Stealth] (5.5,6.5) -- (5.5,11.5);
\draw [->, >=Stealth] (4.75,8) -- (11,8);
\draw [short] (5.5,8) -- (8.25,10.5);
\draw [short] (8.25,10.5) -- (8.25,6.75);
\draw [short] (8.25,6.75) -- (8.75,6.75);
\draw [short] (8.75,6.75) -- (8.75,9.25);
\draw [short] (8.75,9.25) -- (10.25,9.25);
\draw [short] (10.25,9.25) -- (10.25,8);
\draw [dashed] (8.25,6.75) -- (5.5,6.75);
\draw [dashed] (8.75,9.25) -- (5.5,9.25);
\draw [dashed] (8.25,10.5) -- (5.5,10.5);
\node [font=\normalsize] at (5,10.5) {$10V$};
\node [font=\normalsize] at (5,9.25) {$5V$};
\node [font=\normalsize] at (5,6.75) {$-5V$};
\node [font=\normalsize] at (5.25,7.75) {$0$};
\node [font=\normalsize] at (8,7.75) {$10$};
\node [font=\normalsize] at (9,7.75) {$12$};
\node [font=\normalsize] at (10.25,7.75) {$20$};
\node [font=\normalsize] at (11.75,7.5) {$time\brak{ms}$};
\node [font=\normalsize] at (5.5,12) {$v\brak{t}$};
\end{circuitikz}
}%

\label{fig:my_label}
\end{figure}
\begin{multicols}{4}
\begin{enumerate}
\item $4V$
\item $5V$
\item $8V$
\item $10V$
\end{enumerate}
\end{multicols}
}
\item{
The bridge method commonly used for finding mutual inductance is
\begin{multicols}{2}
    \begin{enumerate}
        \item Heavy Campbell Bridge
        \item Schering Bridge
        \item De Sauty bridge
        \item Wien bridge
    \end{enumerate}
\end{multicols}
}
\end{enumerate}
\end{document}

