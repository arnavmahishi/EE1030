\let\negmedspace\undefined
\let\negthickspace\undefined
\documentclass[journal]{IEEEtran}
\usepackage[a5paper, margin=10mm, onecolumn]{geometry}
%\usepackage{lmodern} % Ensure lmodern is loaded for pdflatex
\usepackage{tfrupee} % Include tfrupee package

\setlength{\headheight}{1cm} % Set the height of the header box
\setlength{\headsep}{0mm}     % Set the distance between the header box and the top of the text
\usepackage{xparse}
\usepackage{gvv-book}
\usepackage{gvv}
\usepackage{cite}
\usepackage{amsmath,amssymb,amsfonts,amsthm}
\usepackage{algorithmic}
\usepackage{graphicx}
\usepackage{textcomp}
\usepackage{xcolor}
\usepackage{txfonts}
\usepackage{listings}
\usepackage{enumitem}
\usepackage{mathtools}
\usepackage{gensymb}
\usepackage{comment}
\usepackage[breaklinks=true]{hyperref}
\usepackage{tkz-euclide} 
\usepackage{listings}
% \usepackage{gvv}                                        
\def\inputGnumericTable{} 
\usepackage[latin1]{inputenc}                                
\usepackage{color}                                            
\usepackage{array}                                            
\usepackage{longtable}                                       
\usepackage{calc}                                             
\usepackage{multirow}                                         
\usepackage{hhline}                                           
\usepackage{ifthen}                                           
\usepackage{lscape}

\begin{document}

\bibliographystyle{IEEEtran}
\vspace{3cm}

\title{2019-AE-40-52}
\author{EE24BTECH11006 - Arnav Mahishi}
% \maketitle
% \newpage
% \bigskip
{\let\newpage\relax\maketitle}
\begin{enumerate}
\item{
The airplane shown in figure starts executing a symmetric pull-up maneuver from steady level attitude with a constant nose-up pitch acceleration of $20\frac{deg}{s^2}$. The vertical load factor measured at this instant at the centre of gravity $\brak{\text{CG}}$ is $2$. Given that the acceleration due to gravity is $9.81\frac{m}{s^2}$, the vertical load factor measured at point $P$ on the nose of the airplane, which is $2$ m ahead of the CG, is \rule{3cm}{0.15mm} $\brak{\text{round off to $2$ decimal places}}$.
\begin{figure}[H]
\centering
\resizebox{5cm}{!}{%
\begin{circuitikz}
\tikzstyle{every node}=[font=\normalsize]
\draw [short] (7.25,9.25) .. controls (4.75,9.5) and (4.5,10.25) .. (7.25,10.75);
\draw [short] (7.25,10.75) -- (12,10.25);
\draw [short] (12,10.25) .. controls (14.25,9.75) and (13,9.5) .. (12,9.25);
\draw [short] (11.5,10.25) -- (12,11.5);
\draw [short] (12,11.5) -- (13,11.5);
\draw [short] (13,11.5) -- (13,10);
\draw [short] (12,9.25) -- (7.25,9.25);
\draw [short] (8,9.25) -- (9.5,9.5);
\draw [short] (9.5,9.5) -- (11.25,8.75);
\draw [short] (8,9.25) -- (11.25,8.75);
\draw [short] (12,9.25) -- (13.75,9);
\draw [short] (12.5,9.5) -- (13.75,9);
\draw [<->, >=Stealth] (5.25,8.5) -- (9.5,8.5);
\draw [->, >=Stealth] (5,9.25) .. controls (4.5,9.5) and (4.25,10) .. (5,10.5) ;
\draw  (9,10) circle (0.25cm);
\draw [short] (9,10.25) -- (9,9.75);
\draw [short] (8.75,10) -- (9.25,10);
\node [font=\normalsize] at (4,10) {$20\frac{deg}{s^2}$};
\node [font=\normalsize] at (7.5,8.25) {$2m$};
\node [font=\normalsize] at (9.75,10) {$CG$};
\node [font=\normalsize] at (5.75,10) {$P$};
\draw [short] (5.25,10) -- (5.25,8.25);
\draw [short] (9.5,9.25) -- (9.5,8.5);
\end{circuitikz}
}%

\label{fig:my_label}
\end{figure}
}
\item{
Consider an airplane with a weight of $8000$ N, wing area of $16m^2$, wing zero-lift drag coefficient of $0.02$, Oswald's efficiency factor of $0.8$, and wing aspect ratio of $6$, in steady level flight with wing lift coefficient of $0.375$. Considering the same flight speed and ambient density, the ratio of the induced drag coefficient during steady level flight to that during a $30\degree$ climb is \rule{2cm}{0.15mm}$\brak{\text{round off to $2$ decimal places}}$.\\
\item{
The product of earth's mass $\brak{M}$ and the universal gravitational constant $\brak{G}$ is $GM=3.986\times10^{14}\frac{m^3}{s^2}$. The radius of earth is $6371km$. The minimum increment n the velocity to be imparted to a spacecraft flying in a circular orbit around the earth at an altitude of $4000km$ to make it exit earth's gravitational field is \rule{2cm}{0.15mm}$\frac{km}{s}$ $\brak{\text{round off to $2$ decimal places}}$.\\
}
\item{
A propeller driven airplane has a gross take-off weight of $4905$ N with a wing area of $6.84 m^2$. Assume that the wings are operating at the maximum $\frac{C_L^{\frac{3}{2}}}{C_D}$, of $13$, the propeller efficiency is $0.9$ and the specific fuel consumption of the engine is $0.76\frac{kg}{kW-hr}$. Given that the density of air at sea level is $1.225\frac{kg}{m^3}$ and the acceleration due to gravity is $9.81\frac{m}{s^2}$, the weight of the fuel required for an endurance of $18$ hours at sea level is \rule{2cm}{0.15mm}$\brak{\text{round off to the nearest integer}}$.\\
}
\item{
The design of an airplane is modified to increase the vertical tail area by $20\%$ and decrease the moment arm from the aerodynamic centre of the vertical tail to the airplane centre of gravity by $20\%$. Assuming all other factors remain unchanged, the ratio of the modified to the original directional static stability $\brak{C_{N_\beta}\text{, due to tail fin}}$ is \rule{2cm}{0.15mm}
$\brak{\text{round of to two decimal places}}$\\
}
\item{
For a rocket engine, the velocity ratio $r$ is $\frac{V_a}{V_e}$, where $V_a$ is the vehicle velocity and $V_e$ is the exit velocity of the exhaust gases. Assume the flow to be optimally expanded through the nozzle. For $r=2$, if $F$ is the thrust produced and $m$ is the mass flow rate of exhaust gases, then, $\frac{F}{mV_e}$ is \rule{2cm}{0.15mm}\\
}
\item{
The specific impulse of a rocket engine is $3000\frac{Ns}{kg}$.The mass of the rocket at burnout is $1000$kg. The propellant consumed in the process is $720$ kg. Assume all factors contributing to velocity loss to be negligible.The change in vehicle velocity $\delta u$ is \rule{2cm}{0.15mm} $\frac{km}{s}\brak{\text{ round off to $2$ decimal places}}$.\\
}
\item{
The combustion products of a gas turbine engine can be assumed to be a calorically perfect gas with $\gamma= 1.2$. The pressure ratio across the turbine stage is $0.14$. The measured turbine inlet and exit stagnation temperatures are $1200$ K and $900$ K, respectively. The total-to-total turbine efficiency is \rule{2cm}{0.15mm}$\%$ $\brak{\text{round off to the nearest integer}}$.
}
\item{
The figure shows the velocity triangles for an axial compressor stage. The specific work input to the compressor stage is \rule{2cm}{0.15mm}$\frac{kJ}{kg}\brak{\text{ round of to $2$ decimal places}}$.
\begin{figure}[H]
\centering
\resizebox{5cm}{!}{%
\begin{circuitikz}
\tikzstyle{every node}=[font=\normalsize]
\draw [short] (5.25,6.5) -- (11.5,6.5);
\draw [short] (5.25,6.5) -- (9.5,10);
\draw [short] (9.5,10) -- (11.5,6.5);
\draw [short] (11.5,6.5) -- (6.5,10);
\draw [short] (6.5,10) -- (5.25,6.5);
\draw [short] (9.5,10) -- (9.5,11.25);
\draw [short] (11.5,6.5) -- (11.5,11.25);
\draw [short] (6.5,10) -- (4.25,10);
\draw [<->, >=Stealth] (4.75,10) -- (4.75,6.5);
\draw [<->, >=Stealth] (9.5,10.75) -- (11.5,10.75);
\draw [<->, >=Stealth] (11.5,6.25) -- (5.25,6.25);
\node [font=\normalsize] at (4,8.5) {$c_2=60\frac{m}{s}$};
\node [font=\normalsize] at (10.5,11) {$40\frac{m}{s}$};
\node [font=\normalsize] at (4,10.25) {$\alpha_1=30\degree$};
\node [font=\normalsize] at (8,6) {$U=100\frac{m}{s}$};
\node [font=\normalsize] at (5.75,8.75) {$c_1$};
\draw (6.5,10) to[short] (6.5,8.5);
\node [font=\normalsize] at (6.25,8.5) {$\alpha_1$};
\node [font=\normalsize] at (7,7.75) {$c_2$};
\node [font=\normalsize] at (9.25,7.75) {$w_1$};
\node [font=\normalsize] at (10.75,8.25) {$w_2$};
\draw [->, >=Stealth] (13.25,9.5) -- (13.25,7.5);
\draw [->, >=Stealth] (13.25,9.5) -- (15.25,9.5);
\node [font=\normalsize] at (13,8) {$z$};
\node [font=\normalsize] at (15,10) {$\theta$};
\draw [short] (6.25,9) -- (6.5,9);
\end{circuitikz}
}%
\label{fig:my_label}
\end{figure}
}
\item{
As shown in the figure, a rigid slab CD of weight W $\brak{\text{distributed uniformly along its length}}$ is hung from a ceiling using three cables of identical length and cross-sectional area. The central cable is made of steel $\brak{\text{Young's modulus = 3E}}$ and the other two cables are made of aluminium $\brak{\text{Young's modulus=E}}$. The percentage of the total weight taken by the central cable is \rule{2cm}{0.15mm}$\%\brak{\text{round off to the nearest integer}}$
\begin{figure}[H]
\centering
\resizebox{5cm}{!}{%
\begin{circuitikz}
\tikzstyle{every node}=[font=\normalsize]
\draw [short] (4.25,8.75) -- (4.25,6);
\draw [short] (3.75,8.75) -- (10,8.75);
\draw [short] (9.75,8.75) -- (9.75,6);
\draw [short] (7,8.75) -- (7,6);
\draw [short] (4.25,6) -- (9.75,6);
\draw [short] (9.75,6) -- (9.75,5.5);
\draw [short] (9.75,5.5) -- (4.25,5.5);
\draw [short] (4.25,5.5) -- (4.25,6);
\draw [short] (4.25,9.25) -- (3.75,8.75);
\draw [short] (4.75,9.25) -- (4.25,8.75);
\draw [short] (5.25,9.25) -- (4.75,8.75);
\draw [short] (5.75,9.25) -- (5.25,8.75);
\draw [short] (6.25,9.25) -- (5.75,8.75);
\draw [short] (6.75,9.25) -- (6.25,8.75);
\draw [short] (7.25,9.25) -- (6.75,8.75);
\draw [short] (7.75,9.25) -- (7.25,8.75);
\draw [short] (8.25,9.25) -- (7.75,8.75);
\draw [short] (8.75,9.25) -- (8.25,8.75);
\draw [short] (9.25,9.25) -- (8.75,8.75);
\draw [short] (9.75,9.25) -- (9.25,8.75);
\draw [short] (10.25,9.25) -- (9.75,8.75);
\node [font=\normalsize] at (4,7) {$E$};
\node [font=\normalsize] at (6.5,7) {$3E$};
\node [font=\normalsize] at (10,7) {$E$};
\node [font=\normalsize] at (4,5.5) {$C$};
\node [font=\normalsize] at (10,5.5) {$D$};
\node [font=\normalsize] at (5.5,10.5) {$a$};
\node [font=\normalsize] at (8.5,10.5) {$a$};
\draw [<->, >=Stealth] (3.75,10) -- (7,10);
\draw [<->, >=Stealth] (7,10) -- (9.75,10);
\end{circuitikz}
}%

\label{fig:my_label}
\end{figure}
}
\item{
All the bars in the given truss are elastic with Young's modulus $200$GPa, and have identical cross-sections with moment of inertia $0.1$cm. The lowest value of the load P at which the truss fails due to buckling is \rule{2cm}{0.15mm}KN $\brak{\text{round off to the nearest integer}}$.
\begin{figure}[H]
\centering
\resizebox{5cm}{!}{%
\begin{circuitikz}
\tikzstyle{every node}=[font=\normalsize]
\draw (5.25,8.5) to[short, -o] (7.75,11) ;
\draw [short] (7.75,11) -- (10.25,8.5);
\draw (5.25,8.5) to[short] (10.25,8.5);
\draw [short] (5.25,8.5) -- (4.75,8);
\draw [short] (5.25,8.5) -- (5.75,8);
\draw [short] (4.5,8) -- (6,8);
\draw [short] (10.25,8.5) -- (9.75,8);
\draw [short] (10.25,8.5) -- (10.75,8);
\draw [short] (9.5,8) -- (11,8);
\draw [short] (4.5,8) -- (4.75,7.75);
\draw [short] (4.75,8) -- (5,7.75);
\draw [short] (5,8) -- (5.25,7.75);
\draw [short] (5.25,8) -- (5.5,7.75);
\draw [short] (5.5,8) -- (5.75,7.75);
\draw [short] (9.75,8) -- (10,7.75);
\draw [short] (10,8) -- (10.25,7.75);
\draw [short] (10.25,8) -- (10.5,7.75);
\draw [short] (10.5,8) -- (10.75,7.75);
\draw [short] (9.5,8) -- (9.75,7.75);
\draw [short] (10.75,8) -- (11,7.75);
\draw [short] (7.5,10.75) -- (7.75,10.5);
\draw [short] (7.75,10.5) -- (8,10.75);
\draw [short] (6,9.25) .. controls (6.5,9.25) and (6.75,9) .. (6.5,8.5);
\draw [short] (9.5,9.25) .. controls (9.25,9.25) and (8.75,9.25) .. (9,8.5);
\node [font=\normalsize] at (7,9.25) {$45\degree$};
\node [font=\normalsize] at (8.75,9.25) {$45\degree$};
\draw [<->, >=Stealth] (5.25,7.5) -- (10.5,7.5);
\node [font=\normalsize] at (7.5,7.75) {$10cm$};
\node [font=\normalsize] at (5,8.5) {$B$};
\node [font=\normalsize] at (7.25,11) {$A$};
\node [font=\normalsize] at (10.75,8.5) {$C$};
\node [font=\normalsize] at (8,12) {$P$};
\draw [->, >=Stealth] (7.75,12.25) -- (7.75,11);
\draw [short] (10,8) -- (10,8.25);
\draw [short] (10,8.25) -- (10.5,8.25);
\draw [short] (10.5,8.25) -- (10.5,8);
\end{circuitikz}
}%

\label{fig:my_label}
\end{figure}
}
\item{
A solid circular shaft is designed to transmit a torque T with a factor of safety of $2$. It is proposed to replace the solid shaft by a hollow shaft of the same material and identical outer radius. If the inner radius is half the outer radius, the factor of safety for the hollow shaft is \rule{2cm}{0.15mm}$\brak{\text{round off to $1$ decimal place}}$.
\begin{figure}[H]
\centering
\resizebox{5cm}{!}{%
\begin{circuitikz}
\tikzstyle{every node}=[font=\normalsize]
\draw  (5.5,10.5) circle (1cm);
\draw  (9.25,10.5) circle (1cm);
\draw [->, >=Stealth] (5.5,10.5) -- (6,11.25);
\draw  (9.25,10.5) circle (0.5cm);
\draw [->, >=Stealth] (9.25,10.5) -- (9.75,10.25);
\draw [->, >=Stealth] (9.25,10.5) -- (9.75,11.5);
\draw (6,11.25) to[short] (6.75,11.25);
\draw (9.75,11.5) to[short] (10.75,11.5);
\node [font=\normalsize] at (6.5,11.5) {$R$};
\node [font=\normalsize] at (10.5,11.75) {$R$};
\node [font=\normalsize] at (9.25,10.25) {$\frac{R}{2}$};
\end{circuitikz}
}%

\label{fig:my_label}
\end{figure}
}
\item{
In the structure shown in the figure, bars AB and BC are made of identical material and have circular cross-sections of $10$mm radii. The yield stress of the material under uniaxial tension is $280$MPa. Using the von Mises yield criterion, the maximum load along the $z$-direction $\brak{\text{perpendicular to the plane of paper}}$ that can be applied at C, such that AB does not yield is N $\brak{\text{round off to the nearest integer}}$.\\
\begin{figure}[H]
\centering
\resizebox{5cm}{!}{%
\begin{circuitikz}
\tikzstyle{every node}=[font=\normalsize]
\draw [short] (4.75,11) -- (4.75,8.75);
\draw [short] (4.75,10) -- (9.75,10);
\draw (9.75,10) to[short, -o] (9.75,13.75) ;
\draw [short] (4.75,11) -- (4.5,10.75);
\draw [short] (4.75,10.75) -- (4.5,10.5);
\draw [short] (4.75,10.5) -- (4.5,10.25);
\draw [short] (4.75,10.25) -- (4.5,10);
\draw [short] (4.75,10) -- (4.5,9.75);
\draw [short] (4.75,9.75) -- (4.5,9.5);
\draw [short] (4.75,9.5) -- (4.5,9.25);
\draw [short] (4.75,9.25) -- (4.5,9);
\draw [short] (4.75,9) -- (4.5,8.75);
\draw [short] (4.75,8.75) -- (4.5,8.5);
\draw [->, >=Stealth] (4.75,11) -- (4.75,12);
\draw [->, >=Stealth] (4.75,9.75) -- (6.5,9.75);
\node [font=\normalsize] at (4.75,12.25) {$y$};
\node [font=\normalsize] at (6.5,9.5) {$x$};
\node [font=\normalsize] at (9.5,10.25) {$B$};
\node [font=\normalsize] at (9.25,14) {$C$};
\node [font=\normalsize] at (7,12) {$AB=0.55m\text{    }BC=0.5m$};
\node [font=\normalsize] at (5,10.25) {$A$};
\end{circuitikz}
}%
\label{fig:my_label}
\end{figure}
}}
\end{enumerate}
\end{document}
