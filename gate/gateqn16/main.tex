\let\negmedspace\undefined
\let\negthickspace\undefined
\documentclass[journal]{IEEEtran}
\usepackage[a5paper, margin=10mm, onecolumn]{geometry}
%\usepackage{lmodern} % Ensure lmodern is loaded for pdflatex
\usepackage{tfrupee} % Include tfrupee package

\setlength{\headheight}{1cm} % Set the height of the header box
\setlength{\headsep}{0mm}     % Set the distance between the header box and the top of the text
\usepackage{xparse}
\usepackage{gvv-book}
\usepackage{gvv}
\usepackage{cite}
\usepackage{amsmath,amssymb,amsfonts,amsthm}
\usepackage{algorithmic}
\usepackage{graphicx}
\usepackage{textcomp}
\usepackage{xcolor}
\usepackage{txfonts}
\usepackage{listings}
\usepackage{enumitem}
\usepackage{mathtools}
\usepackage{gensymb}
\usepackage{comment}
\usepackage[breaklinks=true]{hyperref}
\usepackage{tkz-euclide} 
\usepackage{listings}
% \usepackage{gvv}                                        
\def\inputGnumericTable{} 
\usepackage[latin1]{inputenc}                                
\usepackage{color}                                            
\usepackage{array}                                            
\usepackage{longtable}                                       
\usepackage{calc}                                             
\usepackage{multirow}                                         
\usepackage{hhline}                                           
\usepackage{ifthen}                                           
\usepackage{lscape}

\begin{document}

\bibliographystyle{IEEEtran}
\vspace{3cm}

\title{2022-EE-1-13}
\author{EE24BTECH11006 - Arnav Mahishi}
% \maketitle
% \newpage
% \bigskip
{\let\newpage\relax\maketitle}
\begin{enumerate}
\item{
As you grow older, an injury to your \rule{1cm}{0.15mm} may take longer to \rule{1cm}{0.15mm}
\begin{multicols}{4}
\begin{enumerate}
\item heel/heel
\item heal/heel
\item heal/heal 
\item heel/heal
\end{enumerate}
\end{multicols}
}
\item{
In a $500$m race, $P$ and $Q$ have speeds in the ratio of $3:4$. $Q$ starts the race when $P$ has aldready covered $140$m. What is the distance between $P$ and $Q\brak{\text{in m}}$ when $P$ wins the race?
\begin{multicols}{4}
\begin{enumerate}
\item $20$
\item $40$
\item $60$
\item $140$
\end{enumerate}
\end{multicols}}
\item{
Three bells P, Q, and R are rung periodically in a school. P is rung every $20$ minutes; Q is rung every $30$ minutes and R is rung every $50$ minutes.
If all the three bells are rung at $12:00$ PM, when will the three bells ring together again the next time?
\begin{multicols}{4}
\begin{enumerate}
\item $5:00$PM
\item $5:30$PM
\item $6:00$PM
\item $6:30$PM
\end{enumerate}
\end{multicols}
}
\item{
Given below are two statements and four conclusions drawn based on the
statements.\\
Statement $1$: Some bottles are cups.\\
Statement $2$: All cups are knives.\\
Conclusion $I$: Some bottles are knives.\\
Conclusion $II$: Some knives are cups.\\
Conclusion $III$: All cups are bottles.\\
Conclusion $IV$: All knives are cups.\\
Which one of the following options can be logically inferred?
\begin{enumerate}
\item Only conclusion $I$ and conclusion $II$ are correct
\item Only conclusion $II$ and conclusion $III$ are correct
\item Only conclusion $II$ and conclusion $IV$ are correct
\item Only conclusion $III$ and conclusion $IV$ are correct
\end{enumerate}
}
\item{
The figure below shows the front and rear view of a disc, which is shaded with identical patterns. The disc is flipped once with respect to any one of the fixed axes $1-1,2-2$ or $3-3$ chosen uniformly at random.
What is the probability that the disc DOES NOT retain the same front and rear views after the flipping operation?
\begin{figure}[H]
\centering
\resizebox{7cm}{!}{%
\begin{circuitikz}
\tikzstyle{every node}=[font=\normalsize]
\draw  (5.5,10) circle (1.5cm);
\draw  (11,10) circle (1.5cm);
\draw [dashed] (5.5,12) -- (5.5,8);
\draw [dashed] (3.25,10) -- (7.75,10);
\draw [dashed] (4,8.5) -- (7,11.5);
\draw [dashed] (3.75,11.75) -- (7,8.5);
\draw [dashed] (11,12.75) -- (11,7.75);
\draw [dashed] (8.5,10) -- (14.25,10);
\draw [dashed] (13,12) -- (9,8);
\draw [dashed] (8.7,12) -- (13,8);
\node [font=\normalsize] at (5.5,7.25) {Front View};
\node [font=\normalsize] at (11,6.75) {Rear View};
\node [font=\normalsize] at (5.5,12.5) {1};
\node [font=\normalsize] at (3.5,12) {3};
\node [font=\normalsize] at (7,12) {2};
\node [font=\normalsize] at (3.5,8.5) {2};
\node [font=\normalsize] at (7,8) {3};
\node [font=\normalsize] at (8.5,12) {2};
\node [font=\normalsize] at (11,13) {1};
\node [font=\normalsize] at (13,12.5) {3};
\node [font=\normalsize] at (13.25,7.75) {2};
\node [font=\normalsize] at (11,7.25) {1};
\node [font=\normalsize] at (9,7.5) {3};
\draw [short] (4.25,10.75) -- (4.25,10);
\draw [short] (4.5,11) -- (4.5,10);
\draw [short] (4.75,10.75) -- (4.75,10);
\draw [short] (5,10.25) -- (5,10);
\draw [short] (5,10.5) -- (5,10);
\draw [short] (6,10.5) -- (6,10);
\draw [short] (6.25,10.75) -- (6.25,10);
\draw [short] (6.5,11) -- (6.5,10);
\draw [short] (6.75,10.75) -- (6.75,10);
\draw [short] (5,9.5) -- (6,9.5);
\draw [short] (4.75,9.25) -- (6.25,9.25);
\draw [short] (6.25,9.25) -- (4.75,9.25);
\draw [short] (4.5,9) -- (6.5,9);
\draw [short] (4.75,8.75) -- (6.25,8.75);
\draw [short] (9.75,10.75) -- (9.75,10);
\draw [short] (10,10.75) -- (10,10);
\draw [short] (10.25,10.5) -- (10.25,10);
\draw [short] (10.5,10.25) -- (10.5,10);
\draw [short] (11.25,10.5) -- (11.25,10);
\draw [short] (11.5,10.5) -- (11.5,10);
\draw [short] (11.75,10.75) -- (11.75,10);
\draw [short] (12,11) -- (12,10);
\draw [short] (10.75,9.75) -- (11.25,9.75);
\draw [short] (10.5,9.5) -- (11.5,9.5);
\draw [short] (10.25,9.25) -- (11.75,9.25);
\draw [short] (10,9) -- (12,9);
\draw [short] (10.25,8.75) -- (11.75,8.75);
\node [font=\normalsize] at (5.5,8) {1};
\end{circuitikz}
}%

\label{fig:my_label}
\end{figure}
\begin{multicols}{4}
\begin{enumerate}
\item $0$
\item $\frac{1}{3}$
\item $\frac{2}{3}$
\item $1$
\end{enumerate}
\end{multicols}
}
\item{
Altruism is the human concern for the wellbeing of others. Altruism has been shown to be motivated more by social bonding, familiarity and identification of belongingness to a group. The notion that altruism may be attributed to empathy or guilt has now been rejected.
Which one of the following is the CORRECT logical inference based on the information in the above passage?
\begin{enumerate}
\item Humans engage in altruism due to guilt but not empathy
\item Humans engage in altruism due to empathy but not guilt
\item Humans engage in altruism due to group identification but not empathy
\item Humans engage in altruism due to empathy but not familiarty
\end{enumerate}
}
\item{
There are two identical dice with a single letter on each of the faces. The following six letters: Q, R, S, T, U, and V, one on each of the faces. Any of the six outcomes are equally likely.
The two dice are thrown once independently at random.
What is the probability that the outcomes on the dice were composed only of any combination of the following possible outcomes: Q, U and V?
\begin{multicols}{4}
\begin{enumerate}
\item $\frac{1}{4}$
\item $\frac{3}{4}$
\item $\frac{1}{6}$
\item $\frac{5}{36}$
\end{enumerate}
\end{multicols}
}
\item{
The price of an item is $10\%$ cheaper in an online store S compared to the price
at another online store M. Store S charges $Rs.150$ for delivery. There are no
delivery charges for orders from the store M. A person bought the item from the store S and saved $Rs.100$.
What is the price of the item at the online store S $\brak{\text{in Rs.}}$ if there are no other charges than what is described above?
\begin{multicols}{4}
\begin{enumerate}
\item $2500$
\item $2250$
\item $1750$
\item $1500$
\end{enumerate}
\end{multicols}
}
\item{
The letters P, Q, R, S, T and U are to be placed one per vertex on a regular convex hexagon, but not necessarily in the same order.\\
Consider the following statements:\\
$1$.The line segment joining R and S is longer than the line segment joining P and Q.\\
$2$.The line segment joining R and S is perpendicular to the line segment joining P and Q.\\
$3$.The line segment joining R and U is parallel to the line segment joining T and Q.\\
Based on the above statements, which one of the following options is
CORRECT?
\begin{enumerate}
\item The line segment joining R and T is parallel to the line segment joining Q and S
\item The line segment joining T and Q is parallel to the line segment joining P and U
\item The line segment joining R and P is parallel to the line segment joining U and Q
\item The line segment joining Q and S is parallel to the line segment joining R and P
\end{enumerate}
}
\item{
An ant is at the bottom-left corner of a grid $\brak{\text{point P}}$ as shown below. It aims to move to the top-right corner of the grid. The ant moves only along the lines
marked in the grid such that the current distance to the top-right corner strictly decreases.
Which one of the following is a part of a possible trajectory of the ant during the movement?
\begin{figure}[H]
\centering
\resizebox{3cm}{!}{%
\begin{circuitikz}
\tikzstyle{every node}=[font=\normalsize]
\draw  (7,10.75) rectangle (11.75,6.5);
\draw [short] (8.25,10.75) -- (8.25,6.5);
\draw [short] (9.5,10.75) -- (9.5,6.5);
\draw [short] (10.75,10.75) -- (10.75,6.5);
\draw [short] (7,9.75) -- (11.75,9.75);
\draw [short] (11.75,8.5) -- (7,8.5);
\draw [short] (7,7.5) -- (11.75,7.5);
\node [font=\normalsize] at (7,6.25) {P};
\end{circuitikz}
}%
\label{fig:my_label}
\end{figure}
\begin{enumerate}
\item{
\begin{figure}[h!]
\centering
\resizebox{1cm}{!}{%
\begin{circuitikz}
\tikzstyle{every node}=[font=\normalsize]
\draw (7.25,7.25) to[short] (6.5,7.25);
\draw (6.5,7.25) to[short] (6.5,8);
\draw (6.5,8) to[short] (7.25,8);
\draw (7.25,8) to[short] (7.25,8.5);
\draw (7.25,8.5) to[short] (8,8.5);
\draw (8,8.5) to[short] (8,8);
\node [font=\normalsize] at (6.5,7) {P};
\end{circuitikz}
}%
\label{fig:my_label}
\end{figure}
}
\item{
\begin{figure}[!ht]
\centering
\resizebox{1cm}{!}{%
\begin{circuitikz}
\tikzstyle{every node}=[font=\normalsize]
\draw (5.75,9.25) to[short] (5.75,8.25);
\draw (5.75,8.25) to[short] (6.75,8.25);
\draw (6.75,8.25) to[short] (6.75,10.25);
\draw (6.75,10.25) to[short] (7.75,10.25);
\draw (7.75,10.25) to[short] (7.75,9.25);
\node [font=\normalsize] at (5.5,8) {P};
\end{circuitikz}
}%

\label{fig:my_label}
\end{figure}
}
\item{
\begin{figure}[!ht]
\centering
\resizebox{1cm}{!}{%
\begin{circuitikz}
\tikzstyle{every node}=[font=\normalsize]
\draw (6,7.5) to[short] (6,8.5);
\draw (6,8.5) to[short] (7.25,8.5);
\draw (7.25,8.5) to[short] (7.25,9.5);
\draw (7.25,9.5) to[short] (8.5,9.5);
\node [font=\normalsize] at (5.75,7.25) {P};
\end{circuitikz}
}%

\label{fig:my_label}
\end{figure}
}
\item{
\begin{figure}[!ht]
\centering
\resizebox{1cm}{!}{%
\begin{circuitikz}
\tikzstyle{every node}=[font=\normalsize]
\draw  (7.5,10) rectangle (9.25,8.5);
\node [font=\normalsize] at (7.25,8.25) {P};
\end{circuitikz}
}%

\label{fig:my_label}
\end{figure}
}
\end{enumerate}
}
\item{
The transfer function of a real system, $H\brak{s}$, is given as: $H\brak{s}=\frac{As+b}{s^2+Cs+D}$ where $A,B,C$, and $D$ are positive constants. This system cannot operate as 
\begin{multicols}{4}
\begin{enumerate}
\item low pass filter
\item high pass filter
\item band pass filter
\item in integrator
\end{enumerate}
\end{multicols}
}
\item{
FOr an ideal MOSFET biased in saturation, the magnitude of the small signal current gain for a common drain amplifier is
\begin{multicols}{4}
\begin{enumerate}
\item $0$
\item $1$
\item $100$
\item infinite
\end{enumerate}
\end{multicols}
}
\item{
The most commonly used relay, for the protection of an alternator against loss of excitation, is
\begin{multicols}{2}
    \begin{enumerate}
        \item offset Mho relay
        \item over current relay
        \item differential relay
        \item Buchholz relay
    \end{enumerate}
\end{multicols}
}
\end{enumerate}
\end{document}
