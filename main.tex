%iffalse
\let\negmedspace\undefined
\let\negthickspace\undefined
\documentclass[journal,12pt,twocolumn]{IEEEtran}
\usepackage{cite}
\usepackage{amsmath,amssymb,amsfonts,amsthm}
\usepackage{algorithmic}
\usepackage{graphicx}
\usepackage{textcomp}
\usepackage{xcolor}
\usepackage{txfonts}
\usepackage{listings}
\usepackage{enumitem}
\usepackage{mathtools}
\usepackage{gensymb}
\usepackage{comment}
\usepackage[breaklinks=true]{hyperref}
\usepackage{tkz-euclide} 
\usepackage{listings}
\usepackage{gvv}                                        
%\def\inputGnumericTable{}                                 
\usepackage[latin1]{inputenc}                                
\usepackage{color}                                            
\usepackage{array}                                            
\usepackage{longtable}                                       
\usepackage{calc}                                             
\usepackage{multirow}                                         
\usepackage{hhline}                                           
\usepackage{ifthen}                                           
\usepackage{lscape}
\usepackage{tabularx}
\usepackage{array}
\usepackage{float}
\newtheorem{theorem}{Theorem}[section]
\newtheorem{problem}{Problem}
\newtheorem{proposition}{Proposition}[section]
\newtheorem{lemma}{Lemma}[section]
\newtheorem{corollary}[theorem]{Corollary}
\newtheorem{example}{Example}[section]
\newtheorem{definition}[problem]{Definition}
\newcommand{\BEQA}{\begin{eqnarray}}
\newcommand{\EEQA}{\end{eqnarray}}
\newcommand{\define}{\stackrel{\triangle}{=}}
\theoremstyle{remark}
\newtheorem{rem}{Remark}
\usepackage{multicol}
\newcounter{sectioicolsn}

% Marks the beginning of the document
\begin{document}
\bibliographystyle{IEEEtran}
\vspace{3cm}

\title{EE1030: Matrix Theory}
\author{EE24BTECH11006 - Arnav Mahishi}
\maketitle
\newpage
\bigskip

\renewcommand{\thefigure}{\theenumi}
\renewcommand{\thetable}{\theenumi}
F. Match the Following \\
In these questions there are entries in columns 1 and 2. Each entry in column 1 is related to exactly one entry in column 2. Write the correct letter from column 2 against the entry number in column 1 in your answer book
\\
$1. \frac{sin3\alpha}{cos2\alpha} \; is $\hfill{\sbrak{1992-2 Marks}}
\\
\begin{multicols}{2}
Column I
\\
(A) Positive
\\
(B) Negative
\columnbreak
\\
Column II
\\
(p) $\brak{\frac{13\pi}{48},\frac{14\pi}{48}}$
\\
(q) $\brak{\frac{14\pi}{48},\frac{18\pi}{48}}$
\\
(r) $\brak{\frac{18\pi}{48},\frac{23\pi}{48}}$
\\
(s) $\brak{0,\frac{\pi}{2}}$


\end{multicols}


2. Let $f(x)=sin(\pi cosx)$ and $g(x)=cos(2\pi sinx)$ be two functions defined for $x>0$. Define the following sets whose elements are written in the increasing order.\hfill{\sbrak{JEE Adv. 2019}}
\\\\$X=\{x:f(x)=0\}$,$Y=\{x:f'(x)=0\}$\\
$Z=\{x:g(x)=0\}, W=\{x:g'(x)=0\}$
\\
\begin{multicols}{2}
Column I
\\
(A) X
\\
(B) Y
\\
(C) Z
\\
(D) W
\columnbreak
\\
Column II
\\
(p) $\supseteq \left\{ \frac{\pi}{2}, \frac{3\pi}{2}, 4\pi, 7\pi \right\}$
\\
(q)an arithmetic progression
\\
(r)NOT an arithmetic progression
\\
(s)$\supseteq\left\{\frac{\pi}{6},\frac{7\pi}{6},\frac{13\pi}{6}\right\}$


\end{multicols}
Which of the following is the only CORRECT combination?
\\
\begin{enumerate}[label=\alph*]
\item(IV),(P),(R),(S)
\item(III),(P),(Q),(U)
\item(III),(R),(U)
\item(IV),(Q),(T)
\end{enumerate}


3. Let $f(x)=sin(\pi cosx)$ and $g(x)=cos(2\pi sinx)$ be two functions defined for $x>0$. Define the following sets whose elements are written in the increasing order.\hfill{\sbrak{JEE Adv. 2019}}
\\\\$X=\{x:f(x)=0\}$,$Y=\{x:f'(x)=0\}$\\
$Z=\{x:g(x)=0\}, W=\{x:g'(x)=0\}$
\\
\begin{multicols}{2}
Column I
\\
(A) X
\\
(B) Y
\\
(C) Z
\\
(D) W
\columnbreak
\\
Column II
\\
(p) $\supseteq \left\{ \frac{\pi}{2}, \frac{3\pi}{2}, 4\pi, 7\pi \right\}$
\\
(q)an arithmetic progression
\\
(r)NOT an arithmetic progression
\\
(s)$\supseteq\left\{\frac{\pi}{6},\frac{7\pi}{6},\frac{13\pi}{6}\right\}$


\end{multicols}
Which of the following is the only CORRECT combination?
\\
\begin{enumerate}[label=\alph*]
\item(I),(Q),(U)
\item(I),(P),(R)
\item(II),(R),(S)
\item(II),(Q),(T)
\end{enumerate}

Paragraph 1

Let O be the origin, and $\overrightarrow{OX},\overrightarrow{OY},
\overrightarrow{OZ} $ be three unit vectors in the directions of the sides $\overrightarrow{QR},\overrightarrow{RP},\overrightarrow{PQ} $ respectively, of a triangle PQR.\hfill{[JEE Adv 2017]}$\\\\
$1$.\abs{\overrightarrow{OX}\times\overrightarrow{OY}}=$
\\
\begin{enumerate}[label=\alph*]
\item$sin(P+Q)$ 
\item$sin2R$
\item$sin(P+R)$
\item$sin(Q+R)$
\end{enumerate}

2. If the triangle PQR varies, then the minimum value of $cos(P+Q)+cos(Q+R)+cos(R+P)$ is.
\\
\begin{enumerate}[label=\alph*]
\item$\frac{-5}{3}$
\item$\frac{-3}{2}$
\item$\frac{3}{2}$
\item$\frac{5}{3}$
\end{enumerate}



I. Integer value type
\\

1. The number of all possible values of $\theta $ where   $ 0<\theta<\pi $ for which the system of equations$ \\$

$(y+Z)cos3\theta=(xyz)sin3\theta$\\

$xsin3\theta=\frac{2cos3\theta}{y}+\frac{2sin3\theta}{z}$\\

$(xyz)sin3\theta=(y+2z)cos3\theta +ysin3\theta$\\

have a solution ($x_{o},y_{o},x_{o}$) with $y_{o}z_{o}\neq0$, is\hfill{\sbrak{2010}}
\\

$2$. The number of all possible values of $\theta$ in the interval,
$\brak{\frac{-\pi}{2},\frac{\pi}{2}}$  such that $\theta\neq\frac{n\pi}{5} for n=0,\pm1,\pm2 $ and $tan\theta=cot5\theta $ as well as $sin2\theta=cos4\theta$  is \hfill{\sbrak{2010}}
\\

$3$.The maximum value of the expression
\\$\frac{1}{sin^2\theta+3sin\theta cos\theta+5cos^2\theta}$ is  \hfill{[2010]}
\\


$4$. Two parallel chords of a circle of radius 2 are at a distance$    \left(\sqrt{3}+1\right) $\space apart. If the chords subtend at the center, angles of $\frac{\pi}{k} and \frac{2\pi}{k}$, where $k>0$, the value of [k] is \hfill{\sbrak{2010}}
\\

$5$. The positive integer value of $n>3$ satisfying the equation $\frac{1}{sin\left(\frac{\pi}{n}\right)}=\frac{1}{sin\left(\frac{2\pi}{n}\right)}+\frac{1}{sin\left(\frac{3\pi}{n}\right)}$ is\hfill{\sbrak{2010}}
\end{document}


