\let\negmedspace\undefined
\let\negthickspace\undefined
\documentclass[journal]{IEEEtran}
\usepackage[a5paper, margin=10mm, onecolumn]{geometry}
%\usepackage{lmodern} % Ensure lmodern is loaded for pdflatex
\usepackage{tfrupee} % Include tfrupee package

\setlength{\headheight}{1cm} % Set the height of the header box
\setlength{\headsep}{0mm}     % Set the distance between the header box and the top of the text
\usepackage{xparse}
\usepackage{gvv-book}
\usepackage{gvv}
\usepackage{cite}
\usepackage{amsmath,amssymb,amsfonts,amsthm}
\usepackage{algorithmic}
\usepackage{graphicx}
\usepackage{textcomp}
\usepackage{xcolor}
\usepackage{txfonts}
\usepackage{listings}
\usepackage{enumitem}
\usepackage{mathtools}
\usepackage{gensymb}
\usepackage{comment}
\usepackage[breaklinks=true]{hyperref}
\usepackage{tkz-euclide} 
\usepackage{listings}
% \usepackage{gvv}                                        
\def\inputGnumericTable{} 
\usepackage[latin1]{inputenc}                                
\usepackage{color}                                            
\usepackage{array}                                            
\usepackage{longtable}                                       
\usepackage{calc}                                             
\usepackage{multirow}                                         
\usepackage{hhline}                                           
\usepackage{ifthen}                                           
\usepackage{lscape}

\begin{document}

\bibliographystyle{IEEEtran}
\vspace{3cm}

\title{2010-EE-1-13}
\author{EE24BTECH11006 - Arnav Mahishi}
% \maketitle
% \newpage
% \bigskip
{\let\newpage\relax\maketitle}
\begin{enumerate}
\item{
The value of the quantity $P$, where $P=\int_0^1xe^x$ is equal to
\begin{multicols}{4}
\begin{enumerate}
\item $0$
\item $1$
\item $e$ 
\item $\frac{1}{e}$
\end{enumerate}
\end{multicols}
}
\item{
Divergence of the three-dimensional radial vector field $\overrightarrow{r}$
\begin{multicols}{4}
\begin{enumerate}
\item $3$
\item $\frac{1}{r}$
\item $\hat{i}+\hat{j}+\hat{k}$
\item $3\brak{\hat{i}+\hat{j}+\hat{k}}$
\end{enumerate}
\end{multicols}}
\item{
The period of the signal $x\brak{t}=8\sin\brak{0.8t+\frac{\pi}{4}}$ is
\begin{multicols}{4}
\begin{enumerate}
\item $0.4\pi s$
\item $0.8\pi s$
\item $1.25s$
\item $2.5s$
\end{enumerate}
\end{multicols}
}
\item{
The system represented by the input-output relationship $y\brak{t}=\int_{-\infty}^t x\brak{\tau}d\tau,t\text{\textgreater}0$ is
\begin{multicols}{2}
\begin{enumerate}
\item Linear and casual 
\item Linear but non casual 
\item Casual but not linear
\item Neither linear nor casual
\end{enumerate}
\end{multicols}
}
\item{
The switch in the circuit has been closed for a long time. It is opened at $t=0$. At $t=0^+$, the current through the $1\mu F$ capacitor is.
\begin{figure}[H]
\centering
\resizebox{5cm}{!}{
\begin{circuitikz}
\tikzstyle{every node}=[font=\normalsize]
\draw (4.5,10.75) to[battery1] (4.5,9.5);
\draw (4.5,10.75) to[short] (5.25,10.75);
\draw (5.25,10.75) to[R] (6.25,10.75);
\draw (6.25,10.75) to[short] (6.75,10.75);
\draw (6.75,10.75) to[opening switch] (8.75,10.75);
\draw (8.75,10.75) to[C] (8.75,9.5);
\draw (4.5,9.5) to[short] (8.75,9.5);
\draw (8.75,9.5) to[short] (10.5,9.5);
\draw (8.75,10.75) to[short] (10.5,10.75);
\draw (10.5,10.75) to[R] (10.5,9.5);
\node [font=\normalsize] at (3.75,10) {5V};
\node [font=\normalsize] at (5.75,11.25) {1$\Omega$};
\node [font=\normalsize] at (7.75,10.5) {t=0};
\node [font=\normalsize] at (9.5,10) {$1\mu F$};
\node [font=\normalsize] at (11,10) {4$\Omega$};
\end{circuitikz}
}
\end{figure}
\begin{multicols}{4}
\begin{enumerate}
\item $0A$
\item $1A$
\item $1.25A$
\item $5A$
\end{enumerate}
\end{multicols}
}
\item{
The second harmonic component of the periodic waveform given in the figure has an amplitude of
\begin{figure}[H]
\centering
\resizebox{5cm}{!}{%
\begin{circuitikz}
\tikzstyle{every node}=[font=\Huge]
\draw [->, >=Stealth] (10,11) -- (9.75,26);
\draw [->, >=Stealth] (4.25,14.5) -- (28,14.5);
\draw [short] (10,18.25) -- (15.25,18.25);
\draw [short] (15.25,18.25) -- (15.5,10.75);
\draw [shot] (15.5,10.75) -- (21.25,10.75);
\draw [short] (21.25,10.75) -- (21,18.25);
\draw [short] (21,18.25) -- (24.75,18.25);
\draw [short] (10,11.25) -- (10,7.5);
\node [font=\Huge] at (9.25,13.75) {0};
\node [font=\Huge] at (9,18) {1};
\draw (10,10.5) to[short] (4.5,10.5);
\node [font=\Huge] at (9.25,10) {-1};
\node [font=\Huge] at (16,13.5) {$\frac{T}{2}$};
\node [font=\Huge] at (21.75,14) {$T$};
\end{circuitikz}
}%
\end{figure}
\begin{multicols}{4}
\begin{enumerate}
\item Normal
\item Gamma
\item Beta
\item Cauchy
\end{enumerate}
\end{multicols}
}
\item{
As shown in the figure, a resistance $1\Omega$ resistance is connected across a source that has a load line $v+i=100$. The current through the resistance is
\begin{figure}[H]
\centering
\resizebox{3cm}{!}{%
\begin{circuitikz}
\tikzstyle{every node}=[font=\normalsize]
\draw  (5.5,10) rectangle (8.75,8.25);
\draw (8.75,9.75) to[short] (10.75,9.75);
\draw (8.75,8.5) to[short] (10.75,8.5);
\draw [short] (10.75,9.75) -- (10.75,9.5);
\draw [short] (10.75,8.5) -- (10.75,8.75);
\draw  (10.25,9.5) rectangle (11.25,8.75);
\node [font=\normalsize] at (10.75,9) {1$\Omega$};
\draw [->, >=Stealth] (9,10) -- (9.75,10);
\node [font=\normalsize] at (10,10) {i};
\node [font=\normalsize] at (8.5,9.75) {+};
\node [font=\normalsize] at (8.5,8.5) {-};
\node [font=\normalsize] at (8.5,9) {V};
\node [font=\normalsize] at (7,9) {Source};
\end{circuitikz}
}%

\label{fig:my_label}
\end{figure}
\begin{multicols}{4}
\begin{enumerate}
\item $25A$
\item $50A$
\item $100A$
\item $200A$
\end{enumerate}
\end{multicols}
}
\item{
A wattmeter is connected as shown in the figure. The wattmeter reads
\begin{figure}[H]
\centering
\resizebox{5cm}{!}{%
\begin{circuitikz}
\tikzstyle{every node}=[font=\normalsize]
\draw (6.5,9.5) to[sinusoidal voltage source, sources/symbol/rotate=auto] (8.5,9.5);
\draw (6.5,9.5) to[short] (6.5,7.25);
\draw (8.5,9.5) to[short] (9.25,9.5);
\draw  (9.25,9.75) rectangle (11.5,9.25);
\draw (11.5,9.5) to[short] (13.5,9.5);
\draw (13.5,9.5) to[short] (13.5,9);
\draw  (13,9) rectangle (14,8.25);
\draw (13.5,8.25) to[short] (13.5,7.75);
\draw (6.5,7.25) to[short] (6.5,6);
\draw (13.5,7.75) to[short] (13.5,7.25);
\draw  (13,7.25) rectangle (14,6.5);
\draw (6.5,6) to[short] (13.5,6);
\draw (13.5,6.5) to[short] (13.5,6);
\draw (7.75,6) to[short] (7.75,8.25);
\draw (7.75,8.25) to[short] (9.25,8.25);
\draw  (9.25,8.5) rectangle (11.5,8);
\draw (11.5,8.25) to[short] (12.25,8.25);
\draw (12.25,8.25) to[short] (12.25,7.75);
\draw (12.25,7.75) to[short] (13.5,7.75);
\draw [ dashed] (8.75,10.25) rectangle  (12,7.75);
\node [font=\normalsize] at (10.25,9.5) {Current Coil};
\node [font=\normalsize] at (10.25,8.25) {Potential Coil};
\node [font=\normalsize] at (13.5,8.5) {};
\node [font=\normalsize] at (13.5,8.5) {$Z_1$};
\node [font=\normalsize] at (11.5,5) {Text};
\node [font=\normalsize] at (13.5,6.75) {$Z_2$};
\node [font=\normalsize] at (10.25,7.5) {Wattmeter};
\end{circuitikz}
}%

\label{fig:my_label}
\end{figure}
\begin{multicols}{2}
\begin{enumerate}
\item Zero always
\item Total power consumed by $Z_1$ and $Z_2$
\item Power consumed by $Z_1$
\item Power consumed by $Z_2$
\end{enumerate}
\end{multicols}
}
\item{
An ammeter has current range $0-5A$ and its internal resistance is $0.2\Omega$. In order to change the range to $0-25A$, we need to add a resistance of
\begin{multicols}{2}
\begin{enumerate}
\item $0.8\Omega$ in series with the meter.
\item $1.0\Omega$ in series with the meter.
\item $0.04\Omega$ in parallel with the meter.
\item $0.05\Omega$ in parallel with the meter.
\end{enumerate}
\end{multicols}
}
\item{
As shown in the figure, a negative feedback system has an amplifier of gain $100$ with $\pm 10\%$ tolerance in the forward path, and an attenutator of value $\frac{9}{100}$ in the feedback path. The overall system gain is approximately.
\begin{figure}[H]
\centering
\resizebox{5cm}{!}{%
\begin{circuitikz}
\tikzstyle{every node}=[font=\normalsize]
\draw  (5.75,10.25) circle (0.5cm);
\draw [->, >=Stealth] (4.25,10.25) -- (5.25,10.25);
\draw [->, >=Stealth] (6.25,10.25) -- (7.75,10.25);
\draw  (7.75,10.75) rectangle (10.5,10);
\draw [->, >=Stealth] (10.5,10.25) -- (13.5,10.25);
\draw [->, >=Stealth] (12.25,10.25) -- (10,8.25);
\draw  (8.5,8.75) rectangle (10,7.75);
\draw [->, >=Stealth] (8.5,8.25) -- (6,9.75);
\node [font=\normalsize] at (9,10.5) {$100\pm 10\%$};
\node [font=\normalsize] at (9.25,8.25) {$\frac{9}{100}$};
\node [font=\normalsize] at (5,10.5) {+};
\node [font=\normalsize] at (5.75,9.5) {-};
\end{circuitikz}
}%

\label{fig:my_label}
\end{figure}
\begin{multicols}{4}
\begin{enumerate}
\item $10\pm 1\%$
\item $10\pm 2\%$
\item $10\pm 5\%$ 
\item $10\pm 10\%$
\end{enumerate}
\end{multicols}
}
\item{
For the system $\frac{2}{s+1}$, The approximate time taken for a step response to reach $98\%$ of its final value is
\begin{multicols}{4}
\begin{enumerate}
\item $1s$
\item $2s$
\item $4s$
\item $8s$
\end{enumerate}
\end{multicols}
}
\item{
If the electrical circuit of figure$\brak{\text{b}}$ is an equivalent of the coupled tank system of figure$\brak{\text{a}}$, then
\begin{figure}[H]
\centering
\resizebox{10cm}{!}{%
\begin{circuitikz}
\tikzstyle{every node}=[font=\normalsize]
\draw [short] (2.25,12.25) -- (3.75,12.25);
\draw [short] (2.25,11.75) -- (3.25,11.75);
\draw [short] (3.25,11.75) -- (3.25,11.25);
\draw [short] (3.75,12.25) -- (3.75,11.25);
\draw [->, >=Stealth] (3.5,11.5) -- (3.5,11);
\draw [short] (2.75,11.25) -- (2.75,8.75);
\draw [short] (2.75,8.75) -- (6.75,8.75);
\draw [short] (4.25,11.25) -- (4.25,9.5);
\draw [short] (4.25,9.5) -- (4.25,9.25);
\draw [short] (4.25,9.25) -- (5.25,9.25);
\draw [short] (5.25,9.25) -- (5.25,11.25);
\draw [short] (6.75,11.25) -- (6.75,9.25);
\draw [short] (6.75,9.25) -- (7.75,9.25);
\draw [short] (7.75,9.25) -- (7.75,8.25);
\draw [short] (6.75,8.75) -- (7.25,8.75);
\draw [short] (7.25,8.75) -- (7.25,8.25);
\draw [->, >=Stealth] (7.5,8.5) -- (7.5,8);
\draw [dashed] (2.75,10.5) -- (4.25,10.5);
\draw [dashed] (5.25,10) -- (6.75,10);
\draw [->, >=Stealth] (3.5,10) -- (3.5,10.5);
\draw [->, >=Stealth] (3.5,9.25) -- (3.5,8.75);
\draw [->, >=Stealth] (6,9.75) -- (6,10);
\draw [->, >=Stealth] (6,9.25) -- (6,8.75);
\draw [->, >=Stealth] (4.25,9) -- (5.25,9);
\node [font=\normalsize] at (3.5,9.75) {$h_1$};
\node [font=\normalsize] at (6,9.5) {$h_2$};
\draw [ dashed] (2.25,10.75) rectangle  (8,8.5);
\node [font=\normalsize] at (4.5,7.5) {$\brak{\text{a}}$ Coupled Tank};
\draw (10,8.75) to[american current source] (10,10.75);
\draw (10,10.75) to[short] (11.5,10.75);
\draw (10,8.75) to[short] (11.5,8.75);
\draw (11.5,10.75) to[short] (11.5,10.25);
\draw (11.5,8.75) to[short] (11.5,9.25);
\draw  (11.25,10.25) rectangle (11.75,9.25);
\draw (11.5,10.75) to[short] (12,10.75);
\draw  (12,11) rectangle (13,10.5);
\draw (13,10.75) to[short] (13.5,10.75);
\draw (13.5,10.75) to[short] (13.5,10.25);
\draw  (13.25,10.25) rectangle (13.75,9.25);
\draw (13.5,9.25) to[short] (13.5,8.75);
\draw (13.5,10.75) to[short] (14.25,10.75);
\draw (14.25,10.75) to[short] (14.25,10.25);
\draw  (14,10.25) rectangle (14.5,9.25);
\draw (14.25,9.25) to[short] (14.25,8.75);
\draw (11.5,8.75) to[short] (14.25,8.75);
\node [font=\normalsize] at (12.5,10.75) {$B$};
\node [font=\normalsize] at (11.5,10) {$A$};
\node [font=\normalsize] at (13.5,9.75) {$C$};
\node [font=\normalsize] at (14.25,9.75) {$D$};
\node [font=\normalsize] at (12,7.5) {$\brak{\text{b}}$ Electrical Equivalent};
\draw [ dashed] (10.75,11.5) rectangle  (14.75,8.25);
\end{circuitikz}
}%
\end{figure}
\begin{enumerate}
\item $A,B$ are resistances and $C,D$ are capacitances
\item $A,C$ are resistances and $B,D$ are capacitances
\item $A,B$ are resistances and $C,D$ are capacitances
\item $A,C$ are resistances and $B,D$ are capacitances
\end{enumerate}
}
\item{
A single-phase transformer has turns ratio of $1:2$ and is connected to a purely resistive load as shown in the figure. The magentizing current drawn is $1A$ and the secondary current is $1A$. If core losses and leakage reactances are neglected, the primary current is
\begin{figure}[H]
\centering
\resizebox{5cm}{!}{%
\begin{circuitikz}
\tikzstyle{every node}=[font=\normalsize]
\draw (6,13.75) to[sinusoidal voltage source, sources/symbol/rotate=auto] (8,13.75);
\draw (6,13.75) to[short] (6,12.25);
\draw (6,12.25) to[short] (8,12.25);
\draw (8,13.75) to[L ] (8,12.25);
\draw (8.5,13.75) to[short] (8.5,12.25);
\draw (8.75,13.75) to[short] (8.75,12.25);
\draw (9.25,13.75) to[short] (11.75,13.75);
\draw (9.25,12.25) to[short] (11.75,12.25);
\draw (11.75,13.75) to[R] (11.75,12.25);
\draw (9.25,13.75) to[L ] (9.25,12.25);
\node [font=\normalsize] at (8.5,14.25) {$1:2$};
\draw [->, >=Stealth] (9.75,14) -- (10.75,14);
\node [font=\normalsize] at (10.25,14.25) {$1A$};
\end{circuitikz}
}%
\end{figure}
\begin{multicols}{4}
\begin{enumerate}
    \item $1.4A$
    \item $2A$
    \item $2.24A$
    \item $3A$
\end{enumerate}
\end{multicols}

}
\end{enumerate}
\end{document}

