\let\negmedspace\undefined
\let\negthickspace\undefined
\documentclass[journal]{IEEEtran}
\usepackage[a5paper, margin=10mm, onecolumn]{geometry}
%\usepackage{lmodern} % Ensure lmodern is loaded for pdflatex
\usepackage{tfrupee} % Include tfrupee package

\setlength{\headheight}{1cm} % Set the height of the header box
\setlength{\headsep}{0mm}     % Set the distance between the header box and the top of the text
\usepackage{xparse}
\usepackage{gvv-book}
\usepackage{gvv}
\usepackage{cite}
\usepackage{amsmath,amssymb,amsfonts,amsthm}
\usepackage{algorithmic}
\usepackage{graphicx}
\usepackage{textcomp}
\usepackage{xcolor}
\usepackage{txfonts}
\usepackage{listings}
\usepackage{enumitem}
\usepackage{mathtools}
\usepackage{gensymb}
\usepackage{comment}
\usepackage[breaklinks=true]{hyperref}
\usepackage{tkz-euclide} 
\usepackage{listings}
% \usepackage{gvv}                                        
\def\inputGnumericTable{} 
\usepackage[latin1]{inputenc}                                
\usepackage{color}                                            
\usepackage{array}                                            
\usepackage{longtable}                                       
\usepackage{calc}                                             
\usepackage{multirow}                                         
\usepackage{hhline}                                           
\usepackage{ifthen}                                           
\usepackage{lscape}

\begin{document}

\bibliographystyle{IEEEtran}
\vspace{3cm}

\title{3-3.4-6}
\author{EE24BTECH11006 - Arnav Mahishi}
% \maketitle
% \newpage
% \bigskip
{\let\newpage\relax\maketitle}

\renewcommand{\thefigure}{\theenumi}
\renewcommand{\thetable}{\theenumi}
\setlength{\intextsep}{10pt} % Space between text and floats


\numberwithin{equation}{enumi}
\numberwithin{figure}{enumi}
\renewcommand{\thetable}{\theenumi}
Q) Construct a rhombus whose diagonals are 4cm and 6cm in lengths.\\
\begin{table}[h!]    
  \centering
  \begin{tabular}[10pt]{ |c| c| c|}
    \hline
    \textbf{input} & \textbf{value}\\ 
    \hline
    $x_1$&$\myvec{0\\0}$\\
    \hline 
    $x_2$&$\myvec{a\\0}$\\
    \hline 
    $x_3$&$\myvec{0\\b}$\\
    \hline
    \end{tabular}

  \caption{Input Parameters}
\end{table}\\
Soln: Assuming x and y axis as diagonals of rhombus and center as origin\\
\begin{align}
\norm{\overrightarrow{OA}}=\norm{\overrightarrow{OC}}\implies \overrightarrow{OA}=\myvec{3\\0}\text{ and }\overrightarrow{OC}=\myvec{-3\\0}\\
\norm{\overrightarrow{OB}}=\norm{\overrightarrow{OD}}\implies \overrightarrow{OB}=\myvec{0\\2}\text{ and }\overrightarrow{OD}=\myvec{0\\-2}\\
\implies\text{Sidelength}=\overline{AB}=\sqrt{3^2+2^2}=\sqrt{13}\\
\implies\text{Perimeter}=4\cdot\overline{AB}=4\cdot\sqrt{13}
\end{align}
\begin{figure}[h!]
   \centering
   \includegraphics[width=0.7\linewidth]{figs/Figure_1.png}
   \caption{Plot of rhombus}
   \label{stemplot}
\end{figure}
\end{document}
