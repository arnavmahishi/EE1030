\let\negmedspace\undefined
\let\negthickspace\undefined
\documentclass[journal,12pt,twocolumn]{IEEEtran}
\usepackage{cite}
\usepackage{amsmath,amssymb,amsfonts,amsthm}
\usepackage{algorithmic}
\usepackage{graphicx}
\usepackage{textcomp}
\usepackage{xcolor}
\usepackage{txfonts}
\usepackage{listings}
\usepackage{enumitem}
\usepackage{mathtools}
\usepackage{gensymb}
\usepackage{comment}
\usepackage[breaklinks=true]{hyperref}
\usepackage{tkz-euclide} 
\usepackage{listings}
\usepackage{gvv}                                        
%\def\inputGnumericTable{}                                 
\usepackage[latin1]{inputenc}                                
\usepackage{color}                                            
\usepackage{array}                                            
\usepackage{longtable}                                       
\usepackage{calc}                                             
\usepackage{multirow}                                         
\usepackage{hhline}                                           
\usepackage{ifthen}                                           
\usepackage{lscape}
\usepackage{tabularx}
\usepackage{array}
\usepackage{float}


\newtheorem{theorem}{Theorem}[section]
\newtheorem{problem}{Problem}
\newtheorem{proposition}{Proposition}[section]
\newtheorem{lemma}{Lemma}[section]
\newtheorem{corollary}[theorem]{Corollary}
\newtheorem{example}{Example}[section]
\newtheorem{definition}[problem]{Definition}
\newcommand{\BEQA}{\begin{eqnarray}}
\newcommand{\EEQA}{\end{eqnarray}}
\newcommand{\define}{\stackrel{\triangle}{=}}
\theoremstyle{remark}
\newtheorem{rem}{Remark}

% Marks the beginning of the document
\begin{document}
\bibliographystyle{IEEEtran}
\vspace{3cm}

\title{EE1030: Matrix Theory}
\author{EE24BTECH11006 - Arnav Mahishi}
\maketitle
\newpage
\bigskip

\renewcommand{\thefigure}{\theenumi}
\renewcommand{\thetable}{\theenumi}
\textbf{F. Match the Following} \\
In these questions there are entries in columns 1 and 2. Each entry in column 1 is related to exactly one entry in column 2. Write the correct letter from column 2 against the entry number in column 1 in your answer book
\newline\\
$1. \dfrac{sin3\alpha}{cos2\alpha} \; is $\quad\quad\quad\quad\quad\quad\quad\quad[1992-2 Marks]
\newline

\textbf{Column I}  \hspace{4cm}\textbf{Column II}
\newline

(A) positive \hspace{3cm} (p) $\left(\dfrac{13\pi}{48},\dfrac{14\pi}{48}\right)$
\newline

(B) negative \hspace{3cm} (q) $\left(\dfrac{14\pi}{48},\dfrac{18\pi}{48}\right)$
\newline

\hspace{150pt} (r) $\left(\dfrac{18\pi}{48},\dfrac{23\pi}{48}\right)$
\newline

\hspace{150pt} (s) $\left(0,\dfrac{\pi}{2}\right)$
\newline
\newline

2. Let $f(x)=sin(\pi cosx)$ and $g(x)=cos(2\pi sinx)$ be two functions defined for $x>0$. Define the following sets whose elements are written in the increasing order.\hspace{80pt} [JEE Adv. 2019]
\newline\newline$X=\{x:f(x)=0\}$,$Y=\{x:f'(x)=0\}$\newline
$Z=\{x:g(x)=0\}, W=\{x:g'(x)=0\}$
\newline

\textbf{Column I}  \hspace{2cm}\textbf{Column II}
\newline

(A) X \hspace{42pt} (p) $\supseteq \left\{ \dfrac{\pi}{2}, \dfrac{3\pi}{2}, 4\pi, 7\pi \right\}$


(B) Y \hspace{42pt} (q)an arithmetic progression
\newline

(C) Z \hspace{45pt} (r)NOT an arithmetic progression
\newline

(D) W\hspace{45pt}  (s) $\supseteq\left\{\dfrac{\pi}{6},\dfrac{7\pi}{6},\dfrac{13\pi}{6}\right\}$
\newline

\hspace{75pt} (t) $\supseteq\left\{\dfrac{\pi}{3},\dfrac{2\pi}{3},\pi\right\}$
\newline

Which of the following is the only CORRECT combination?
\newline
(a) (IV),(P),(R),(S) \hspace{50pt} (b) (III),(P),(Q),(U)\newline
(c) (III),(R),(U) \hspace{65pt} (d) (IV),(Q),(T)
\newline

3. Let $f(x)=sin(\pi cosx)$ and $g(x)=cos(2\pi sinx)$ be two functions defined for $x>0$. Define the following sets whose elements are written in the increasing order.\hspace{80pt} [JEE Adv. 2019]
\newline\newline$X=\{x:f(x)=0\}$,$Y=\{x:f'(x)=0\}$\newline
$Z=\{x:g(x)=0\}, W=\{x:g'(x)=0\}$
\newline

\textbf{Column I}  \hspace{2cm}\textbf{Column II}
\newline

(A) X \hspace{42pt} (p)  $\supseteq\left\{\dfrac{\pi}{2},\dfrac{3\pi}{2},4\pi,7\pi\right\}$
\newline

(B) Y \hspace{42pt} (q)an arithmetic progression
\newline

(C) Z \hspace{45pt} (r)NOT an arithmetic progression
\newline

(D) W\hspace{45pt}  (s) $\supseteq\left\{\dfrac{\pi}{6},\dfrac{7\pi}{6},\dfrac{13\pi}{6}\right\}$
\newline

\hspace{75pt} (t) $\supseteq\left\{\dfrac{\pi}{3},\dfrac{2\pi}{3},\pi\right\}$
\newline
Which of the following is the only CORRECT combination?
\newline
(a) (I),(Q),(U) \quad \quad      (b) (I),(P),(R)\newline
(c) (II),(R),(S) \quad \quad (d) (II),(Q),(T)
\newline
\\
\textbf{Paragraph 1}\\
Let O be the origin, and $\overrightarrow{OX},\overrightarrow{OY},
\overrightarrow{OZ} $ be three unit vectors in the directions of the sides $\overrightarrow{QR},\overrightarrow{RP},\overrightarrow{PQ} $ respectively, of a triangle PQR.$\hspace{140pt}$[JEE Adv 2017]$\newline\\
$1$.\abs{\overrightarrow{OX}\times\overrightarrow{OY}}=$
\newline

(a) $sin(P+Q)$ \hspace{50pt} (b) $sin2R$\newline

(c) $sin(P+R)$ \hspace{50pt} (d) $sin(Q+R)$
\newline
\\\\\\
2. If the triangle PQR varies, then the minimum value of $cos(P+Q)+cos(Q+R)+cos(R+P)$ is.
\newline

(a) $\dfrac{-5}{3} \hspace{50pt}$ (b) $\dfrac{-3}{2}$\newline

(c) $\dfrac{3}{2} \hspace{58pt}$ (d) $\dfrac{5}{3}$\newline



\textbf{I. Integer value type}
\newline

1. The number of all possible values of $\theta $ where   $ 0<\theta<\pi $ for which the system of equations$ \newline$

$(y+Z)cos3\theta=(xyz)sin3\theta$\newline

$xsin3\theta=\dfrac{2cos3\theta}{y}+\dfrac{2sin3\theta}{z}$\newline

$(xyz)sin3\theta=(y+2z)cos3\theta +ysin3\theta$\newline

have a solution ($x_{o},y_{o},x_{o}$) with $y_{o}z_{o}\neq0$, is\space [2010]
\newline

2. The number of all possible values of $\theta$ in the interval,
$\left(\dfrac{-\pi}{2},\dfrac{\pi}{2}\right)$  such that $\theta\neq\dfrac{n\pi}{5} for n=0,\pm1,\pm2 $ and $tan\theta=cot5\theta $ as well as $sin2\theta=cos4\theta$  is   [2010] 
\newline

3.The maximum value of the expression
\newline$\dfrac{1}{sin^2\theta+3sin\theta cos\theta+5cos^2\theta}$ is\hspace{15pt}  [2010]
\newline


4. Two parallel chords of a circle of radius 2 are at a distance$    \left(\sqrt{3}+1\right) $\space apart. If the chords subtend at the center, angles of $\dfrac{\pi}{k} and \dfrac{2\pi}{k}$, where $k>0$, the value of [k] is \hspace{5cm}[2010]
\newline

5. The positive integer value of $n>3$ satisfying the equation $\dfrac{1}{sin\left(\dfrac{\pi}{n}\right)}=\dfrac{1}{sin\left(\dfrac{2\pi}{n}\right)}+\dfrac{1}{sin\left(\dfrac{3\pi}{n}\right)}$ is\space [2011]
\end{document}
