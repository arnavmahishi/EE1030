\let\negmedspace\undefined
\let\negthickspace\undefined
\documentclass[journal]{IEEEtran}
\usepackage[a5paper, margin=10mm, onecolumn]{geometry}
%\usepackage{lmodern} % Ensure lmodern is loaded for pdflatex
\usepackage{tfrupee} % Include tfrupee package

\setlength{\headheight}{1cm} % Set the height of the header box
\setlength{\headsep}{0mm}     % Set the distance between the header box and the top of the text

\usepackage{gvv-book}
\usepackage{gvv}
\usepackage{cite}
\usepackage{amsmath,amssymb,amsfonts,amsthm}
\usepackage{algorithmic}
\usepackage{graphicx}
\usepackage{textcomp}
\usepackage{xcolor}
\usepackage{txfonts}
\usepackage{listings}
\usepackage{enumitem}
\usepackage{mathtools}
\usepackage{gensymb}
\usepackage{comment}
\usepackage[breaklinks=true]{hyperref}
\usepackage{tkz-euclide} 
\usepackage{listings}
% \usepackage{gvv}                                        
\def\inputGnumericTable{} 
\usepackage[latin1]{inputenc}                                
\usepackage{color}                                            
\usepackage{array}                                            
\usepackage{longtable}                                       
\usepackage{calc}                                             
\usepackage{multirow}                                         
\usepackage{hhline}                                           
\usepackage{ifthen}                                           
\usepackage{lscape}
\begin{document}

\bibliographystyle{IEEEtran}
\vspace{3cm}

\title{1-1.4-9j}
\author{EE24BTECH11006 - Arnav Mahishi}
% \maketitle
% \newpage
% \bigskip
{\let\newpage\relax\maketitle}

\renewcommand{\thefigure}{\theenumi}
\renewcommand{\thetable}{\theenumi}
\setlength{\intextsep}{10pt} % Space between text and floats


\numberwithin{equation}{enumi}
\numberwithin{figure}{enumi}
\renewcommand{\thetable}{\theenumi}

Q) Point $P\brak{5,-3}$ is one of the points of trisection of line segment joining the points A$\brak{7,-2}$ and B$\brak{1,-5}$
\\Soln: $\overline{AQ}=\overline{QR}=\overline{RB}=\frac{1}{3}\overline{AB}$
\begin{table}[h!]    
  \centering
  \begin{tabular}[10pt]{ |c| c| c|}
    \hline
    \textbf{input} & \textbf{value}\\ 
    \hline
    $x_1$&$\myvec{0\\0}$\\
    \hline 
    $x_2$&$\myvec{a\\0}$\\
    \hline 
    $x_3$&$\myvec{0\\b}$\\
    \hline
    \end{tabular}

  \caption{Input Parameters}
\end{table}

\begin{align}
O&=\frac{1}{1+\frac{1}{n}}\brak{M+\frac{1}{2}N}\\
\implies Q&=\frac{1}{1+\frac{1}{2}}\brak{A+\frac{1}{2}B}=
\frac{2}{3}\brak{\myvec{7\\-2}+\frac{1}{2}\myvec{1\\-5}}=\myvec{5\\-3}\\
\implies R&=\frac{1}{1+\frac{1}{2}}\brak{B+\frac{1}{2}A}=
\frac{2}{3}\brak{\myvec{1\\-5}+\frac{1}{2}\myvec{7\\-2}}=\myvec{3\\-4}
\end{align}
$Q=P\brak{5,-3}$ so $\overline{AP}=\overline{PR}=\overline{RB}=\frac{1}{3}\overline{AB}$  \\
$\therefore P$ is one of the two points that trisects the line segment $\overline{AB}$
\begin{figure}[h!]
   \centering
   \includegraphics[width=0.7\linewidth]{figs/Figure_1.png}
   \caption{Plot of trisection}
   \label{stemplot}
\end{figure}


\end{document}
