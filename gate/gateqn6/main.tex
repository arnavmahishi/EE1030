\let\negmedspace\undefined
\let\negthickspace\undefined
\documentclass[journal]{IEEEtran}
\usepackage[a5paper, margin=10mm, onecolumn]{geometry}
%\usepackage{lmodern} % Ensure lmodern is loaded for pdflatex
\usepackage{tfrupee} % Include tfrupee package

\setlength{\headheight}{1cm} % Set the height of the header box
\setlength{\headsep}{0mm}     % Set the distance between the header box and the top of the text
\usepackage{xparse}
\usepackage{gvv-book}
\usepackage{gvv}
\usepackage{cite}
\usepackage{amsmath,amssymb,amsfonts,amsthm}
\usepackage{algorithmic}
\usepackage{graphicx}
\usepackage{textcomp}
\usepackage{xcolor}
\usepackage{txfonts}
\usepackage{listings}
\usepackage{enumitem}
\usepackage{mathtools}
\usepackage{gensymb}
\usepackage{comment}
\usepackage[breaklinks=true]{hyperref}
\usepackage{tkz-euclide} 
\usepackage{listings}
% \usepackage{gvv}                                        
\def\inputGnumericTable{} 
\usepackage[latin1]{inputenc}                                
\usepackage{color}                                            
\usepackage{array}                                            
\usepackage{longtable}                                       
\usepackage{calc}                                             
\usepackage{multirow}                                         
\usepackage{hhline}                                           
\usepackage{ifthen}                                           
\usepackage{lscape}

\begin{document}

\bibliographystyle{IEEEtran}
\vspace{3cm}

\title{2010-XE-14-26}
\author{EE24BTECH11006 - Arnav Mahishi}
% \maketitle
% \newpage
% \bigskip
{\let\newpage\relax\maketitle}
\begin{enumerate}
\item{
The residue of the function $f\brak{z}=\frac{\sin^4z}{\brak{z+\frac{\pi}{4}}^3}$ at $z=\frac{-\pi}{4}$
\begin{multicols}{4}
\begin{enumerate}
\item $2$
\item $1$
\item $-1$ 
\item $-2$
\end{enumerate}
\end{multicols}
}
\item{
The variance of the number of heads resulting from independent tosses of a fair coin is
\begin{multicols}{2}
\begin{enumerate}
\item $\frac{5}{4}$ 
\item $\frac{5}{2}$
\item $\frac{3}{4}$
\item $\frac{3}{2}$
\end{enumerate}
\end{multicols}}
\item{
If the quadrature rule $\int_0^3f\brak{x}dx=\alpha f\brak{1}+\beta f\brak{3}$ is exact for all polynomials of degree $2$ or less, then
\begin{multicols}{4}
\begin{enumerate}
\item $\alpha=\frac{3}{4},\beta=\frac{3}{4}$
\item $\alpha=\frac{3}{4},\beta=\frac{9}{4}$
\item $\alpha=\frac{9}{4},\beta=\frac{3}{4}$
\item $\alpha=\frac{9}{4},\beta=\frac{9}{4}$
\end{enumerate}
\end{multicols}
}
\item{
Given that $\frac{dy}{dx}=1+y^2,y\brak{0}=0$, which one of the following is nearest to $y\brak{0.4}$ computed by Euler's method with step size of $0.2$?
\begin{multicols}{4}
\begin{enumerate}
\item $0.408$
\item $0.404$
\item $0.208$
\item $0.204$
\end{enumerate}
\end{multicols}
}
\item{
$f\brak{x}=$
$\begin{cases}
\frac{\sin x}{x} \text{ if }x\neq 0\\
1\text{ if }x=0
\end{cases}$
Then
\begin{enumerate}
\item $f$ is not continuous at $x=0$
\item $f$ is continuous at $x=0$ but not differentiable at $x=0$
\item $f$ is differentiable at $x=0$ and $f^{\prime}\brak{0}=0$
\item $f$ is differentiable at $x=0$ and $f^{\prime}\brak{0}=1$
\end{enumerate}
}
\item{
Let $u\brak{x,y}=\tan\cbrak{xy\brak{x+y}}$. Then
\begin{enumerate}
\item $x\frac{\delta u}{\delta x}+y\frac{\delta u}{\delta x}=\frac{1}{3}xy\brak{x+y}\sec^2\cbrak{xy\brak{x+y}}$
\item $x\frac{\delta u}{\delta x}-y\frac{\delta u}{\delta x}=\frac{1}{3}xy\brak{x+y}\sec^2\cbrak{xy\brak{x+y}}$
\item $x\frac{\delta u}{\delta x}+y\frac{\delta u}{\delta x}=3xy\brak{x+y}\sec^2\cbrak{xy\brak{x+y}}$
\item $x\frac{\delta u}{\delta x}-y\frac{\delta u}{\delta x}=3xy\brak{x+y}\sec^2\cbrak{xy\brak{x+y}}$
\end{enumerate}
}
\item{
Which one of the following is a particular solution of ordinary differntial equation $x\frac{d^2y}{dx^2}-\frac{dy}{dx}=2x^2f\brak{x}$?
\begin{multicols}{2}
\begin{enumerate}
\item $x^2\int xf\brak{x}dx+\int x^3f\brak{x}dx$
\item $x^2\int f\brak{x}dx+\int x^2f\brak{x}dx$
\item $x^2\int xf\brak{x}dx-\int x^3f\brak{x}dx$
\item $x^2\int f\brak{x}dx-\int x^2f\brak{x}dx$
\end{enumerate}
\end{multicols}
}
\item{
Which one of the following is a possible solution to the partial differential equation $\frac{\delta^2u}{\delta t^2}-\frac{\delta^2u}{\delta x^2}=0$ with boundary conditions $u\brak{0,t}=0,\frac{\delta u\brak{\pi,t}}{\delta x}=0$ for $t\geq 0$,  $u\brak{x,0}=0,\frac{\delta u\brak{x,0}}{\delta t}=\pi$, for $0\leq x\leq \pi$?
\begin{enumerate}
\item $u\brak{x,t}=\sum_{n=0}^\infty a_n\sin\brak{\brak{n+\frac{1}{2}}t}\sin\brak{\brak{n+\frac{1}{2}}x}$
\item $u\brak{x,t}=\sum_{n=0}^\infty a_n\cos\brak{\brak{n+\frac{1}{2}}t}\sin\brak{\brak{n+\frac{1}{2}}x}$
\item $u\brak{x,t}=\sum_{n=0}^\infty a_n\sin\brak{\brak{n+\frac{1}{2}}t}\cos\brak{\brak{n+\frac{1}{2}}x}$
\item $u\brak{x,t}=\sum_{n=0}^\infty a_n\cos\brak{\brak{n+\frac{1}{2}}t}\cos\brak{\brak{n+\frac{1}{2}}x}$
\end{enumerate}
}
\item{
A cylindrical container is filled with a liquid up to half of its height. The container is mounted on the center of a turn-table and is held fixed using a spindle. The turn-table is now rotated about its central axis with a certain angular velocity. After some time interval, the fluid attains rigid body rotation. Which of the following profiles best represents the constant pressure surfaces in the container?
\begin{multicols}{2}
\begin{enumerate}
\item{\begin{figure}[H]
\centering
\resizebox{1cm}{!}{%
\begin{circuitikz}
\tikzstyle{every node}=[font=\LARGE]
\draw  (6.75,11.25) rectangle (9.5,7.75);
\draw [dashed] (8.25,11.75) -- (8.25,7.5);
\draw [short] (6.75,10) .. controls (8.25,8.5) and (8,8) .. (9.5,10);
\draw [short] (6.75,9.5) .. controls (8.25,7.75) and (8.25,7.75) .. (9.5,9.5);
\draw [short] (6.75,8.75) .. controls (7.75,8) and (8.25,7) .. (9.5,8.75);
\draw  (8,7.75) rectangle (8.5,7.5);
\draw [->, >=Stealth] (7.75,7.25) -- (8.75,7.25);
\end{circuitikz}
}%

\label{fig:my_label}
\end{figure}}
\item{
\begin{figure}[H]
\centering
\resizebox{1cm}{!}{%
\begin{circuitikz}
\tikzstyle{every node}=[font=\LARGE]
\draw  (6.75,11.25) rectangle (9.5,7.75);
\draw [dashed] (8.25,11.75) -- (8.25,7.5);
\draw [short] (6.75,10) .. controls (8.25,8.5) and (8,8) .. (9.5,10);
\draw  (8,7.75) rectangle (8.5,7.5);
\draw [->, >=Stealth] (7.75,7.25) -- (8.75,7.25);
\draw [short] (6.75,8.5) -- (9.5,8.5);
\draw [short] (6.75,8.25) -- (9.5,8.25);
\end{circuitikz}
}%

\label{fig:my_label}
\end{figure}
}
\item{\begin{figure}[H]
\centering
\resizebox{1cm}{!}{%
\begin{circuitikz}
\tikzstyle{every node}=[font=\LARGE]
\draw  (6.75,11.25) rectangle (9.5,7.75);
\draw [dashed] (8.25,11.75) -- (8.25,7.5);
\draw  (8,7.75) rectangle (8.5,7.5);
\draw [->, >=Stealth] (7.75,7.25) -- (8.75,7.25);
\draw [short] (6.75,8.75) .. controls (8.25,9.75) and (8.25,9.5) .. (9.5,8.75);
\draw [short] (6.75,8.25) .. controls (8.25,9.25) and (8.25,9.25) .. (9.5,8.25);
\draw [short] (6.75,9) .. controls (8,10.25) and (8.25,10.25) .. (9.5,9);
\end{circuitikz}
}%

\label{fig:my_label}
\end{figure}}
\item{\begin{figure}[H]
\centering
\resizebox{1cm}{!}{%
\begin{circuitikz}
\tikzstyle{every node}=[font=\LARGE]
\draw  (6.75,11.25) rectangle (9.5,7.75);
\draw [dashed] (8.25,11.75) -- (8.25,7.5);
\draw  (8,7.75) rectangle (8.5,7.5);
\draw [->, >=Stealth] (7.75,7.25) -- (8.75,7.25);
\draw [short] (6.75,8.5) -- (9.5,8.5);
\draw [short] (6.75,8.25) -- (9.5,8.25);
\draw [short] (6.75,9) -- (9.5,9);
\end{circuitikz}
}%

\label{fig:my_label}
\end{figure}}
\end{enumerate}
\end{multicols}
}
\item{
Match the items given in the following two columns using appropriate combinations:
\begin{table}[h!]    
  \centering
  \begin{tabular}[10pt]{ |c| c| c|}
    \hline
    \textbf{input} & \textbf{value}\\ 
    \hline
    $x_1$&$\myvec{0\\0}$\\
    \hline 
    $x_2$&$\myvec{a\\0}$\\
    \hline 
    $x_3$&$\myvec{0\\b}$\\
    \hline
    \end{tabular}

\end{table}
\begin{multicols}{2}
\begin{enumerate}
\item $P-1;R-2;Q-3;S-4$
\item $P-1;Q-2;R-3;S-4$
\item $P-1;R-2;S-3;Q-4$ 
\item $P-1;S-2;Q-3;R-4$
\end{enumerate}
\end{multicols}
}
\item{
In the context of boundry layers, which of the following statements is FALSE?
\begin{enumerate}
\item It is a frictional layer, close to the body
\item It is a region where fluid flow is irrotational 
\item It is a region across which the pressure gradient is negligible 
\item It is diffusion layer of vorticity
\end{enumerate}
}
\item{
Consider an ideal fluid flow past a circular cylinder shown in the figure below. The peripheral velocity at a point $P$ on the surface of the cylinder is
\begin{figure}[H]
\centering
\resizebox{5cm}{!}{%
\begin{circuitikz}
\tikzstyle{every node}=[font=\normalsize]
\draw  (7.75,10.25) circle (1.5cm);
\draw [->, >=Stealth] (7.75,10.25) -- (7,11.5);
\draw [->, >=Stealth] (7.75,10.25) -- (6.25,10.25);
\node [font=\normalsize] at (7.25,10.5) {$\theta$};
\node [font=\normalsize] at (6.75,11.75) {P};
\draw [->, >=Stealth] (3.25,12) -- (6,12);
\draw [->, >=Stealth] (3.25,11) -- (6,11);
\draw [->, >=Stealth] (3.25,9.75) -- (5.75,9.75);
\draw [->, >=Stealth] (3.25,8.5) -- (6,8.5);
\node [font=\normalsize] at (4.5,12.5) {$U_{\infty}$};
\end{circuitikz}
}%

\label{fig:my_label}
\end{figure}
\begin{multicols}{4}
\begin{enumerate}
\item $0$
\item $U_\infty$
\item $U_\infty\sin\theta$
\item $2U_\infty\sin\theta$
\end{enumerate}
\end{multicols}
}
\item{
The Rheological diagram depicting the relation between shear stress and strain rate for different types of fluids is shown in the figure below.
\begin{figure}[H]
\centering
\resizebox{5cm}{!}{%
\begin{circuitikz}
\tikzstyle{every node}=[font=\normalsize]
\draw [short] (5.5,12.25) -- (5.5,6.75);
\draw [short] (5.5,6.75) -- (11.25,6.75);
\draw [short] (5.5,9.75) -- (9,11.75);
\draw [short] (5.5,6.75) -- (10.5,10);
\draw [short] (5.5,6.75) .. controls (7,9.75) and (7.75,10.5) .. (9.75,10.5);
\draw [short] (5.5,6.75) .. controls (9,7.25) and (9.75,7.5) .. (11,8.75);
\node [font=\normalsize] at (8.5,12) {P};
\node [font=\normalsize] at (9.5,11) {Q};
\node [font=\normalsize] at (10,10) {R};
\node [font=\normalsize] at (10.5,9) {S};
\node [font=\normalsize, rotate around={90:(0,0)}] at (5,9.5) {Shear Stress};
\node [font=\normalsize] at (8.5,6.5) {Strain Rate};
\end{circuitikz}
}%

\label{fig:my_label}
\end{figure}
The most suitable relation for flow of tooth paste being squeezed out of the tube is given by the curve
\begin{multicols}{4}
    \begin{enumerate}
        \item P
        \item Q
        \item R
        \item S
    \end{enumerate}
\end{multicols}

}
\end{enumerate}
\end{document}

