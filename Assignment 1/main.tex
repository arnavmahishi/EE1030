%iffalse
\let\negmedspace\undefined
\let\negthickspace\undefined
\documentclass[journal,12pt,twocolumn]{IEEEtran}
\usepackage{cite}
\usepackage{amsmath,amssymb,amsfonts,amsthm}
\usepackage{algorithmic}
\usepackage{graphicx}
\usepackage{textcomp}
\usepackage{xcolor}
\usepackage{txfonts}
\usepackage{listings}
\usepackage{enumitem}
\usepackage{mathtools}
\usepackage{gensymb}
\usepackage{comment}
\usepackage[breaklinks=true]{hyperref}
\usepackage{tkz-euclide} 
\usepackage{listings}
\usepackage{gvv}                                        
%\def\inputGnumericTable{}                                 
\usepackage[latin1]{inputenc}                                
\usepackage{color}                                            
\usepackage{array}                                            
\usepackage{longtable}                                       
\usepackage{calc}                                             
\usepackage{multirow}                                         
\usepackage{hhline}                                           
\usepackage{ifthen}                                           
\usepackage{lscape}
\usepackage{tabularx}
\usepackage{array}
\usepackage{float}
\newtheorem{theorem}{Theorem}[section]
\newtheorem{problem}{Problem}
\newtheorem{proposition}{Proposition}[section]
\newtheorem{lemma}{Lemma}[section]
\newtheorem{corollary}[theorem]{Corollary}
\newtheorem{example}{Example}[section]
\newtheorem{definition}[problem]{Definition}
\newcommand{\BEQA}{\begin{eqnarray}}
\newcommand{\EEQA}{\end{eqnarray}}
\newcommand{\define}{\stackrel{\triangle}{=}}
\theoremstyle{remark}
\newtheorem{rem}{Remark}
\usepackage{multicol}
\newcounter{sectioicolsn}

% Marks the beginning of the document
\begin{document}
\bibliographystyle{IEEEtran}
\vspace{3cm}

\title{EE1030: Matrix Theory}
\author{EE24BTECH11006 - Arnav Mahishi}
\maketitle
\newpage
\bigskip

\renewcommand{\thefigure}{\theenumi}
\renewcommand{\thetable}{\theenumi}
F. Match the Following \\
In these questions there are entries in columns $1$ and $2$. Each entry in column $1$ is related to exactly one entry in column $2$. Write the correct letter from column $2$ against the entry number in column $1$ in your answer book
\begin{enumerate}
 \item $\frac{\sin3\alpha}{\cos2\alpha}$ \; is $\hfill{\sbrak{1992-2 Marks}}$
\\
\begin{multicols}{2}
Column I
\\
\brak{A} Positive
\\
\brak{B} Negative
\columnbreak
\\
Column II
\\
\brak{p} $\brak{\frac{13\pi}{48},\frac{14\pi}{48}}$
\\
\brak{q} $\brak{\frac{14\pi}{48},\frac{18\pi}{48}}$
\\
\brak{r} $\brak{\frac{18\pi}{48},\frac{23\pi}{48}}$
\\
\brak{s} $\brak{0,\frac{\pi}{2}}$


\end{multicols}


\item.Let $f\brak{x} = \sin\brak{\pi \cos x}$ and $g\brak{x} = \cos\brak{2\pi \sin x}$ be two functions defined for $x > 0$. Define the following sets whose elements are written in the increasing order. \hfill{\sbrak{JEE Adv. 2019}}
\begin{align}
X=\{x:f\brak{x}=0\},Y=\{x:f'\brak{x}=0\}\\
Z=\{x:g\brak{x}=0\}, W=\{x:g'\brak{x}=0\}
\end{align}
\\
\begin{multicols}{2}
Column I
\\
\brak{A} X
\\
\brak{B} Y
\\
\brak{C} Z
\\
\brak{D} W
\columnbreak
\\
Column II
\\
\brak{p} $\supseteq \left\{ \frac{\pi}{2}, \frac{3\pi}{2}, 4\pi, 7\pi \right\}$
\\
\brak{q}an arithmetic progression
\\
\brak{r}NOT an arithmetic progression
\\
\brak{s}$\supseteq\left\{\frac{\pi}{6},\frac{7\pi}{6},\frac{13\pi}{6}\right\}$


\end{multicols}
Which of the following is the only CORRECT combination?
\begin{enumerate}[label=\alph*]
\item\brak{IV},\brak{P},\brak{R},\brak{S}
\item\brak{III},\brak{P},\brak{Q},\brak{U}
\item\brak{III},\brak{R},\brak{U}
\item\brak{IV},\brak{Q},\brak{T}
\end{enumerate}


\item Let $f\brak{x} = \sin\brak{\pi \cos x}$ and $g\brak{x} = \cos\brak{2\pi \sin x}$ be two functions defined for $x > 0$. Define the following sets whose elements are written in the increasing order. \hfill{\sbrak{JEE Adv. 2019}}
\begin{align}
X=\{x:f\brak{x}=0\},Y=\{x:f'\brak{x}=0\}\\
Z=\{x:g\brak{x}=0\}, W=\{x:g'\brak{x}=0\}
\end{align}
\\
\begin{multicols}{2}
Column I
\\
\brak{A} X
\\
\brak{B} Y
\\
\brak{C} Z
\\
\brak{D} W
\columnbreak
\\
Column II
\\
\brak{p} $\supseteq \left\{ \frac{\pi}{2}, \frac{3\pi}{2}, 4\pi, 7\pi \right\}$
\\
\brak{q}an arithmetic progression
\\
\brak{r}NOT an arithmetic progression
\\
\brak{s}$\supseteq\left\{\frac{\pi}{6},\frac{7\pi}{6},\frac{13\pi}{6}\right\}$


\end{multicols}
Which of the following is the only CORRECT combination?
\begin{enumerate}[label=\alph*]
\item\brak{I},\brak{Q},\brak{U}
\item\brak{I},\brak{P},\brak{R}
\item\brak{II},\brak{R},\brak{S}
\item\brak{II},\brak{Q},\brak{T}
\end{enumerate}
\end{enumerate}
Paragraph 1

Let O be the origin, and $\overrightarrow{OX},\overrightarrow{OY},
\overrightarrow{OZ} $ be three unit vectors in the directions of the sides $\overrightarrow{QR},\overrightarrow{RP},\overrightarrow{PQ} $ respectively, of a triangle PQR.\hfill{\sbrak{JEE Adv 2017}}$\\\\
$1$.\abs{\overrightarrow{OX}\times\overrightarrow{OY}}=$
\begin{enumerate}[label=\alph*]
\item$\sin\brak{P+Q}$ 
\item$\sin2R$
\item$\sin\brak{P+R}$
\item$\sin\brak{Q+R}$
\end{enumerate}

$2$. If the triangle PQR varies, then the minimum value of $\cos\brak{P+Q}+\cos\brak{Q+R}+\cos\brak{R+P}$ is.
\begin{enumerate}[label=\alph*]
\item$\frac{-5}{3}$
\item$\frac{-3}{2}$
\item$\frac{3}{2}$
\item$\frac{5}{3}$
\end{enumerate}



I. Integer value type
\\

$1$. The number of all possible values of $\theta $ where   $ 0<\theta<\pi $ for which the system of equations$ \\$
\begin{enumerate}
\item $\brak{y+Z}\cos3\theta=\brak{xyz}\sin3\theta$\\

\item $x\sin3\theta=\frac{2\cos3\theta}{y}+\frac{2\sin3\theta}{z}$\\

\item $\brak{xyz}\sin3\theta=\brak{y+2z}\cos3\theta +y\sin3\theta$\\
\end{enumerate}



have a solution $\brak{x_{o},y_{o},x_{o}}$ with $y_{o}z_{o}$$\neq0$ is \hfill\brak{2010}
\\

$2$. The number of all possible values of $\theta$ in the interval,
$\brak{\frac{-\pi}{2},\frac{\pi}{2}}$  such that $\theta\neq\frac{n\pi}{5} for n=0,\pm1,\pm2 $ and $\tan\theta=\cot5\theta $ as well as $\sin2\theta=\cos4\theta$  is \hfill{\sbrak{2010}}
\\

$3$.The maximum value of the expression
\\$\frac{1}{\sin^2\theta+3\sin\theta \cos\theta+5\cos^2\theta}$ is  \hfill{\sbrak{2010}}
\\


$4$. Two parallel chords of a circle of radius $2$ are at a distance$\brak{\sqrt{3}+1} $\space apart. If the chords subtend at the center, angles of $\frac{\pi}{k} and \frac{2\pi}{k}$, where $k>0$, the value of \sbrak{k} is \hfill{\sbrak{2010}}
\\

$5$. The positive integer value of $n>3$ satisfying the equation $\frac{1}{sin\brak{\frac{\pi}{n}}}=\frac{1}{sin\brak{\frac{2\pi}{n}}}+\frac{1}{sin\brak{\frac{3\pi}{n}}}$ is\hfill{\sbrak{2010}}
\end{document}


