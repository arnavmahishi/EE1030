%iffalse
\let\negmedspace\undefined
\let\negthickspace\undefined
\documentclass[journal,12pt,twocolumn]{IEEEtran}
\usepackage{cite}
\usepackage{amsmath,amssymb,amsfonts,amsthm}
\usepackage{algorithmic}
\usepackage{graphicx}
\usepackage{textcomp}
\usepackage{xcolor}
\usepackage{txfonts}
\usepackage{listings}
\usepackage{enumitem}
\usepackage{mathtools}
\usepackage{gensymb}
\usepackage{comment}
\usepackage[breaklinks=true]{hyperref}
\usepackage{tkz-euclide} 
\usepackage{listings}
\usepackage{gvv}                                        
%\def\inputGnumericTable{}                                 
\usepackage[latin1]{inputenc}                                
\usepackage{color}                                            
\usepackage{array}                                            
\usepackage{longtable}                                       
\usepackage{calc}                                             
\usepackage{multirow}                                         
\usepackage{hhline}                                           
\usepackage{ifthen}                                           
\usepackage{lscape}
\usepackage{tabularx}
\usepackage{array}
\usepackage{float}


\newtheorem{theorem}{Theorem}[section]
\newtheorem{problem}{Problem}
\newtheorem{proposition}{Proposition}[section]
\newtheorem{lemma}{Lemma}[section]
\newtheorem{corollary}[theorem]{Corollary}
\newtheorem{example}{Example}[section]
\newtheorem{definition}[problem]{Definition}
\newcommand{\BEQA}{\begin{eqnarray}}
\newcommand{\EEQA}{\end{eqnarray}}
\newcommand{\define}{\stackrel{\triangle}{=}}
\theoremstyle{remark}
\newtheorem{rem}{Remark}

% Marks the beginning of the document
\begin{document}
\bibliographystyle{IEEEtran}
\vspace{3cm}

\title{Straight Lines}
\author{ee24btech11006-Arnav Mahishi}
\maketitle
\newpage
\bigskip

\renewcommand{\thefigure}{\theenumi}
\renewcommand{\thetable}{\theenumi}
C. Mcq with one correct answer\\

$6$. If the sum of the distances of a point from two perpendicular lines in a plane is 1, then its locus is \hfill{\brak{1992}}
\begin{enumerate}[label=\alph*]
    \item Square
    \item Straight Line
    \item Circle
    \item Two intersecting lines
\end{enumerate}

$7$. The locus of a variable point whose distance from $\brak{-2,0}$ is 
$\frac{2}{3}$ times its distance from the line x=$\frac{-9}{2}$ \hfill{\brak{1994}}
\begin{enumerate}[label=\alph*]
    \item ellipse
    \item hyperbola
    \item parabola
    \item None of these
\end{enumerate}

$8$. The equations to a pair of opposite sides of parallelogram are $x^{2}-5x+6=0$ and $y^{2}-6y+5$, the equations to its diagonals are \hfill{\brak{1994}}
\begin{enumerate}[label=\alph*]
    \item $x+4y=13,y=4x-7$
    \item $4x+y=13,4y=x-7$
    \item $4x+y=13,y=4x-7$
    \item $y-4x=13,y+4x=7$
\end{enumerate}
$9$. The orthocentre of the triangle formed by the lines $xy=0$ and $x+y=1$ is \hfill{\brak{1994}}
\begin{enumerate}[label=\alph*]
    \item $\brak{\frac{1}{2},\frac{1}{2}}$
    \item $\brak{\frac{1}{3},\frac{1}{3}}$
    \item $\brak{0,0}$
    \item $\brak{\frac{1}{4},\frac{1}{4}}$
\end{enumerate}
$10$. Let $PQR$ be a right angled isosceles triangle, right angled at P$\brak{2,1}$.If the equation of the line $QR$ is $2x+y=3$, then the equation representing the pair of lines $PQ$ and $PR$ is \hfill{\brak{1999}
\begin{enumerate}[label=\alph*]
    \item $3x^{2}-3y^{2}+8xy+20x+10y+25=0$
    \item $3x^{2}-3y^{2}+8xy-20x-10y+25=0$
    \item $3x^{2}-3y^{2}+8xy+10x+15y+20=0$
    \item $3x^{2}-3y^{2}-8xy-10x-15y-20=0$
\end{enumerate}
$11.$If $x_1,x_2,x_3$ as well as $y_1,y_2,y_3$ are in G.P with the same common ratio, then the points $\brak{x_1,y_1},\brak{x_2,y_2},\brak{x_3,y_3}$\hfill{\brak{1999}}
\begin{enumerate}[label=\alph*]
    \item lie on a straight line
    \item lie on an ellipse
    \item lie on a circle 
    \item are vertices of a triangle
\end{enumerate}
$12.$ Let $PS$ be the median of the triangle with vertices $P\brak{2,2}$,$Q\brak{6,-1}$ and $R\brak{7,3}$.The equation of the line passing through $\brak{1,1}$ and parallel to $PS$ is \hfill{\brak{2000S}}
\begin{enumerate}[label=\alph*]
    \item $2x-9y-7=0$
    \item $2x-9y-11=0$
    \item $2x+9y-11=0$
    \item $2x+9y+7=0$

\end{enumerate}
$13.$ The incentre of the triangle with vertices $\brak{1,\sqrt{3}},\brak{0,0},and \brak{2,0}$ is \hfill{\brak{2000S}}
\begin{enumerate}[label=\alph*]
    \item $\brak{1,\frac{\sqrt{3}}{2}}$
    \item $\brak{\frac{2}{3},\frac{1}{\sqrt{3}}}$
    \item $\brak{\frac{2}{3},\frac{\sqrt{3}}{2}}$
    \item $\brak{1,\frac{1}{\sqrt{3}}}$
\end{enumerate}
$14$. The number of integer values for $m$, for which the $x$-coordinate of the point of intersection of the lines $3x+4y=9$ and $y=mx+1$ is also an integer,is \hfill{\brak{2001S}}
\begin{enumerate}[label=\alph*]
    \item 2
    \item 0
    \item 4
    \item 1
\end{enumerate}
$15$. Area of the parallelogram formed by the lines $y=mx$,$y=mx+1$,$y=nx$, and $y=nx+1$ equals \hfill{\brak{2001S}}
\begin{enumerate}[label=\alph*]
    \item $\frac{\abs{m+n}}{\brak{m-n}^{2}}$
    \item $\frac{2}{\abs{m+n}}$
    \item $\frac{1}{\abs{m+n}}$
    \item $\frac{1}{\abs{m-n}}$
\end{enumerate}
$16$. Let $0<\alpha<\frac{\pi}{2}$ be fixed angle if $P=\brak{\cos\theta,\sin\theta}$ and $Q=\brak{\cos\brak{\alpha-\theta},\sin\brak{\alpha-\theta}}$, then $Q$ is obtained from $P$ by \hfill{\brak{2002S}}
\begin{enumerate}[label=\alph*]
    \item clockwise rotation around origin through an angle $\alpha$
    \item anticlockwise rotation around origin through an angle $\alpha$
    \item reflection in the line through origin with slope $\tan\alpha$
    \item reflection in the line through origin with slope $\tan\brak{\frac{\alpha}{2}}$
\end{enumerate}
$17$. Let $P=\brak{-1,0},Q=\brak{0,0}$, and $R=\brak{3,3\sqrt{3}}$ be three points. Then the equation of the bisector of the angle $PQR$ is. \hfill{\brak{2002S}}
\begin{enumerate}[label=\alph*]
    \item $\frac{\sqrt{3}}{2}x+y=0$
    \item $x+\sqrt{3}y=0$
    \item $\sqrt{3}x+y=0$
    \item $x+\frac{\sqrt{3}}{2}y=0$
\end{enumerate}
$18$. A straight line through the origin $O$ meets the parallel lines $4x+2y=9$ and $2x+y+6=0$ at points $P$ and $Q$ respectively. Then the point $O$ divides the segment $PQ$ in the ratio \hfill{\brak{2002S}}
\begin{enumerate}[label=\alph*] 
    \item $1\colon2$
    \item $3\colon4$
    \item $2\colon1$
    \item $4\colon3$
\end{enumerate}
$19$. The number of integral points $\brak{integral\ points\ means\ both\ the\ coordinates\ should\ be\ integer}$ exactly in the interior of the triangle with vertices $\brak{0,0},\brak{0,21}$, and $\brak{21,0}$ is \hfill{\brak{2003S}}
\begin{enumerate}[label=\alph*]
    \item $133$
    \item $190$
    \item $233$
    \item $105$
\end{enumerate}
$20$. Orthocentre of triangle with vertices $\brak{0,0},\brak{3,4}$, and $\brak{4,0}$ is \hfill{\brak{2003S}}
\begin{enumerate}[label=\alph*]
    \item $\brak{3,\frac{5}{4}}$
    \item $\brak{3,12}$
    \item $\brak{3,\frac{3}{4}}$
    \item $\brak{3,9}$
\end{enumerate}
\end{document}
