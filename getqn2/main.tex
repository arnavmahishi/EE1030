\let\negmedspace\undefined
\let\negthickspace\undefined
\documentclass[journal]{IEEEtran}
\usepackage[a5paper, margin=10mm, onecolumn]{geometry}
%\usepackage{lmodern} % Ensure lmodern is loaded for pdflatex
\usepackage{tfrupee} % Include tfrupee package

\setlength{\headheight}{1cm} % Set the height of the header box
\setlength{\headsep}{0mm}     % Set the distance between the header box and the top of the text
\usepackage{xparse}
\usepackage{gvv-book}
\usepackage{gvv}
\usepackage{cite}
\usepackage{amsmath,amssymb,amsfonts,amsthm}
\usepackage{algorithmic}
\usepackage{graphicx}
\usepackage{textcomp}
\usepackage{xcolor}
\usepackage{txfonts}
\usepackage{listings}
\usepackage{enumitem}
\usepackage{mathtools}
\usepackage{gensymb}
\usepackage{comment}
\usepackage[breaklinks=true]{hyperref}
\usepackage{tkz-euclide} 
\usepackage{listings}
% \usepackage{gvv}                                        
\def\inputGnumericTable{} 
\usepackage[latin1]{inputenc}                                
\usepackage{color}                                            
\usepackage{array}                                            
\usepackage{longtable}                                       
\usepackage{calc}                                             
\usepackage{multirow}                                         
\usepackage{hhline}                                           
\usepackage{ifthen}                                           
\usepackage{lscape}

\begin{document}

\bibliographystyle{IEEEtran}
\vspace{3cm}

\title{1-1.9-26}
\author{EE24BTECH11006 - Arnav Mahishi}
% \maketitle
% \newpage
% \bigskip
{\let\newpage\relax\maketitle}

\renewcommand{\thefigure}{\theenumi}
\renewcommand{\thetable}{\theenumi}
\setlength{\intextsep}{10pt} % Space between text and floats


\numberwithin{equation}{enumi}
\numberwithin{figure}{enumi}
\renewcommand{\thetable}{\theenumi}
Q) Find the value of $k$, if the point $P\brak{2,4}$ is equidistant from the points $A(5,k)$ and $B(k,7)$.\\

\begin{table}[h!]    
  \centering
  \begin{tabular}[10pt]{ |c| c| c|}
    \hline
    \textbf{input} & \textbf{value}\\ 
    \hline
    $x_1$&$\myvec{0\\0}$\\
    \hline 
    $x_2$&$\myvec{a\\0}$\\
    \hline 
    $x_3$&$\myvec{0\\b}$\\
    \hline
    \end{tabular}

  \caption{Input Parameters}
\end{table}
\begin{align}
\|AP\|=\|PB\| \implies
\brak{A-P}^T\brak{A-P}=\brak{B-P}^T\brak{B-P}\\
\myvec{3\\k-4}^T\myvec{3\\k-4}=\myvec{k-2\\3}^T\myvec{k-2\\3}\\
\myvec{3&k-4}\myvec{3\\k-4}=\myvec{k-2&3}\myvec{k-2\\3}\\
\brak{k-4}^2+9=\brak{k-2}^2+9\\
\brak{k-4}=\pm\brak{k-2}\\
\therefore k=3
\end{align}
\begin{figure}[h!]
   \centering
   \includegraphics[width=0.7\linewidth]{figs/Figure_1.png}
   \caption{Plot of points}
   \label{stemplot}
\end{figure}

\end{document}
