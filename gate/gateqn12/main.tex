\let\negmedspace\undefined
\let\negthickspace\undefined
\documentclass[journal]{IEEEtran}
\usepackage[a5paper, margin=10mm, onecolumn]{geometry}
%\usepackage{lmodern} % Ensure lmodern is loaded for pdflatex
\usepackage{tfrupee} % Include tfrupee package

\setlength{\headheight}{1cm} % Set the height of the header box
\setlength{\headsep}{0mm}     % Set the distance between the header box and the top of the text
\usepackage{xparse}
\usepackage{gvv-book}
\usepackage{gvv}
\usepackage{cite}
\usepackage{amsmath,amssymb,amsfonts,amsthm}
\usepackage{algorithmic}
\usepackage{graphicx}
\usepackage{textcomp}
\usepackage{xcolor}
\usepackage{txfonts}
\usepackage{listings}
\usepackage{enumitem}
\usepackage{mathtools}
\usepackage{gensymb}
\usepackage{comment}
\usepackage[breaklinks=true]{hyperref}
\usepackage{tkz-euclide} 
\usepackage{listings}
% \usepackage{gvv}                                        
\def\inputGnumericTable{} 
\usepackage[latin1]{inputenc}                                
\usepackage{color}                                            
\usepackage{array}                                            
\usepackage{longtable}                                       
\usepackage{calc}                                             
\usepackage{multirow}                                         
\usepackage{hhline}                                           
\usepackage{ifthen}                                           
\usepackage{lscape}

\begin{document}

\bibliographystyle{IEEEtran}
\vspace{3cm}

\title{2017-ME-40-52}
\author{EE24BTECH11006 - Arnav Mahishi}
% \maketitle
% \newpage
% \bigskip
{\let\newpage\relax\maketitle}
\begin{enumerate}
\item{A point mass of $100$ kg is dropped onto a massless elastic bar $\brak{\text{cross-sectional area = $100mm^2$, length = $1$ m, Young's modulus = $100$ GPa}}$ from a height H of $10$ mm as shown \brak{\text{Figure is not to scale}}. If g = $10\frac{m}{s^2}$, the maximum compression of the elastic bar is \rule{3cm}{0.15mm} mm.
\begin{figure}[H]
\centering
\resizebox{3cm}{!}{%
\begin{circuitikz}
\tikzstyle{every node}=[font=\normalsize]
\draw  (6.5,12) circle (0.25cm);
\draw  (6.25,10.5) rectangle (7,6.5);
\draw (5.5,6.5) to[short] (7.75,6.5);
\draw [<->, >=Stealth] (7.75,10.5) -- (7.75,6.5);
\draw [<->, >=Stealth] (7.75,12) -- (7.75,10.5);
\draw [->, >=Stealth] (5.5,11) -- (5.5,9.25);
\node [font=\normalsize] at (5.5,12) {$m=100kg$};
\node [font=\normalsize] at (8.5,11.5) {$H=10mm$};
\node [font=\normalsize] at (8.25,8.5) {$L=1m$};
\node [font=\normalsize] at (5,10) {$g$};
\end{circuitikz}
}%

\label{fig:my_label}
\end{figure}
}
\item{

Two disks $A$ and $B$ with identical mass $\brak{m}$ and radius $\brak{R}$ are initially at rest. They roll down from the top of identical inclined planes without slipping. Disk $A$ has all of its mass concentrated at the rim, while Disk $B$ has its mass uniformly distributed. At the bottom of the plane, the ratio of velocity of the center of disk A to the velocity of the center of disk $B$ is
\begin{multicols}{4}
\begin{enumerate}
\item $\sqrt{\frac{3}{4}}$
\item $\sqrt{\frac{3}{2}}$
\item $1$
\item $\sqrt{2}$
\end{enumerate}
\end{multicols}
}
\item{
A rectangular region in a solid is in a state of plane strain. The $\brak{x,y}$ coordinates of the corners of the undeformed rectangle are given by $P\brak{0,0}$,$Q\brak{4,0}$,$R\brak{4,3}$, $S\brak{0,3}$. The rectangle is subjected to uniform strains:$\epsilon_{xx} = 0.001$, $\epsilon_{yy} = 0.002$, $\gamma_{xy} = 0.003$. The deformed length of the elongated diagonal, up to three decimal places, is \rule{2cm}{0.15mm} units.
}
\item{
A machine element has an ultimate strength $\brak{\sigma_u}$ of $600\frac{N}{mm^2}$ and an endurance limit $\brak{\sigma_{en}}$ of $250\frac{N}{mm^2}$. The fatigue curve for the element on a log-log plot is shown below. If the element is to be designed for a finite life of $10000$ cycles, the maximum amplitude of a completely reversed operating stress is \rule{2cm}{0.15mm} $\frac{N}{mm^2}$ 
 is $\frac{N}{mm^2}$.
 \begin{figure}[H]
\centering
\resizebox{5cm}{!}{%rak{
\begin{circuitikz}
\tikzstyle{every ode}=[font=\normalsize]
\draw [->, >=Stealth] (5.25,7) -- (5.25,12);
\draw [->, >=Stealth] (5.25,7) -- (10.5,7);
\draw [dashed] (5.25,8.75) -- (7,8.75);
\draw [dashed] (7,8.75) -- (7,7);
\draw [short] (5.25,10.25) -- (7,8.75);
\draw [short] (7,8.75) -- (9.25,8.75);
\node [font=\normalsize] at (4.5,10.25) {A};
\node [font=\normalsize] at (5.75,10.5) {$0.8\sigma_u$};
\node [font=\normalsize] at (4.75,8.75) {$\sigma_{en}$};
\node [font=\normalsize] at (5,6.5) {$10^3$};
\node [font=\normalsize] at (7,6.5) {$10^6$};
\node [font=\normalsize] at (7.25,9) {B};
\node [font=\normalsize] at (7.5,5.75) {No. of cycles};
\node [font=\normalsize, rotate around={90:(0,0)}] at (3.5,9) {Faliure Stress};
\end{circuitikz}
}%

\label{fig:my_label}
\end{figure}
}
\item{
A horizontal bar, fixed at one end $\brak{x = 0}$, has a length of $1$ m and a cross-sectional area of $100mm^2$. Its elastic modulus varies along its length as given by $E\brak{x}=100e^{-x}GPa$, where $x$ is the length coordinate in m along the axis of the bar. An axial tensile load of $10$ kN is applied at the free end $\brak{x=1}$. The axial displacement of the free end is \rule{2cm}{0.15mm} mm.}
\item{
In an epicyclic gear train, shown in the figure, the outer ring gear is fixed, while the sun gear rotates counterclockwise at $100$ rpm. Let the number of teeth on the sun, planet and outer gears to be $50, 25$, and $100$, respectively. The ratio of magnitudes of angular velocity of the planet gear to the angular velocity of the carrier arm is \rule{3cm}{0.15mm}
\begin{figure}[H]
\centering
\resizebox{5cm}{!}{%
\begin{circuitikz}
\tikzstyle{every node}=[font=\large]
\draw  (7.25,9) circle (3.25cm);
\draw  (7.25,9) circle (1.75cm);
\draw  (9,10.75) circle (0.75cm);
\draw [short] (7.25,9) -- (9,10.75);
\draw  (7.25,9) circle (0.25cm);
\draw [->, >=Stealth] (3.75,11.25) -- (5.75,10);
\draw [->, >=Stealth] (6.25,14) -- (6.5,12.25);
\draw [->, >=Stealth] (9.5,13) -- (8.75,11.5);
\draw [->, >=Stealth] (12,6.75) -- (8,9.75);
\node [font=\large] at (3,11.5) {Sun Gear};
\node [font=\large] at (6,14.5) {Outer ring gear};
\node [font=\large] at (9.5,13.5) {Planet Gear};
\node [font=\large] at (12.75,6.5) {Carrier arm};
\end{circuitikz}
}%

\label{fig:my_label}
\end{figure}}
\item{

A thin uniform rigid bar of length $L$ and mass $M$ is hinged at point $O$, located at a distance of $\frac{L}{3}$ from one of its ends. The bar is further supported using springs, each of stiffness $k$, located at the two ends. A particle of mass $m =\frac{M}{4}$. is fixed at one end of the bar, as shown in the figure. For small rotations of the bar about $O$, the natural frequency of the system is
\begin{figure}[H]
\centering
\resizebox{5cm}{!}{%
\begin{circuitikz}
\tikzstyle{every node}=[font=\large]
\draw [short] (3.25,10.25) -- (13.25,10.25);
\draw [short] (6,9) -- (7,10.25);
\draw [short] (7,10.25) -- (7.75,9);
\draw [short] (6,9) -- (7.75,9);
\draw [short] (3.25,10.25) -- (5.25,10.25);
\draw (3.25,10.25) to[L ] (3.25,8.5);
\draw (13.25,10.25) to[L ] (13.25,8.5);
\draw [short] (3.25,8.5) -- (2.75,8.5);
\draw [short] (2.75,8.5) -- (2.5,8.25);
\draw [short] (3.25,8.5) -- (3,8.25);
\draw [short] (3.25,8.5) -- (3.5,8.5);
\draw [short] (3.5,8.5) -- (3.75,8.5);
\draw [short] (3.75,8.5) -- (3.5,8.25);
\draw [short] (13.25,8.5) -- (12.5,8.5);
\draw [short] (12.5,8.5) -- (12.25,8.25);
\draw [short] (13.25,8.5) -- (13,8.25);
\draw [short] (13.25,8.5) -- (13.75,8.5);
\draw [short] (13.75,8.5) -- (13.5,8.25);
\draw [<->, >=Stealth] (3.25,10.5) -- (7,10.5);
\draw [<->, >=Stealth] (3.25,11.75) -- (13.25,11.75);
\draw [short] (3.25,10.25) -- (3.25,12.25);
\draw [short] (13.25,10.25) -- (13.25,12.25);
\node [font=\large] at (7.25,10.5) {$O$};
\draw  (12.75,10.75) rectangle (13.25,10.25);
\node [font=\large] at (8,12.25) {$L$};
\node [font=\large] at (5,11) {$\frac{L}{3}$};
\node [font=\large] at (12.25,10.75) {$M$};
\node [font=\large] at (12.75,9.25) {$k$};
\node [font=\large] at (2.75,9.25) {$k$};
\end{circuitikz}
}%

\label{fig:my_label}
\end{figure}
\begin{multicols}{4}
\begin{enumerate}
\item $\sqrt{\frac{5k}{m}}$
\item $\sqrt{\frac{5k}{2m}}$
\item $\sqrt{\frac{3k}{2m}}$
\item $\sqrt{\frac{3k}{m}}$
\end{enumerate}
\end{multicols}
}
\item{
For an inline slider-crank mechanism, the lengths of the crank and connecting rod are $3$ m and $4$ m, respectively. At the instant when the connecting rod is perpendicular to the crank, if the velocity of the slider is $1\frac{m}{s}$, the magnitude of angular velocity $\brak{\text{upto 3 decimal points accuracy}}$ of the crank is \rule{2cm}{0.15mm}$\frac{rad}{s}$.\\
}
\item{
A $10$ mm deep cylindrical cup with diameter of $15$ mm is drawn from a circular blank. Neglecting the variation in the sheet thickness, the diameter $\brak{\text{upto 2 decimal points accuracy}}$ of the blank is \rule{2cm}{0.15mm}mm.\\
}
\item{
Circular arc on a part profile is being machined on a vertical CNC milling machine. CNC part program using metric units with abosolute dimensions is listed below:
\rule{5cm}{0.15mm}\\
N$60$ G$01$ X $30$ Y $55$ Z-$5$ F$50$\\
N$70$ G$02$ X $50$ Y $35$ R $20$\\
N$80$ G$01$ Z $5$\\
\rule{5cm}{0.15mm}
The coordinates of the circular arc are:
\begin{multicols}{4}
    \begin{enumerate}
        \item $\brak{30,55}$
        \item $\brak{50,55}$
        \item $\brak{50,35}$
        \item $\brak{30,35} $
    \end{enumerate}
\end{multicols}
}
\item{
Assume that the surface roughness profile is triangular as shown schematically in the figure. If the peak to valley height is $20\mu m$. The central line average surface roughness $R_a\brak{\text{in }\mu m}$
\begin{figure}[H]
\centering
\resizebox{5cm}{!}{%
\begin{circuitikz}
\tikzstyle{every node}=[font=\large]
\draw [short] (4.25,8) -- (5.25,10);
\draw [short] (5.25,10) -- (6,8);
\draw [short] (6,8) -- (7,10);
\draw [short] (7,10) -- (7.75,8);
\draw [short] (7.75,8) -- (8.5,10);
\draw [short] (8.5,10) -- (9.25,8);
\draw [short] (9.25,8) -- (10,10);
\draw [short] (10,10) -- (10.75,8);
\draw [short] (10.75,8) -- (11.5,10);
\draw [short] (11.5,10) -- (12.5,8);
\end{circuitikz}
}%

\label{fig:my_label}
\end{figure}
\begin{multicols}{4}
    \begin{enumerate}
        \item $5$
        \item $6.67$
        \item $10$
        \item $20$
    \end{enumerate}
\end{multicols}
}
\item{
Two models, P and Q, of a product earn profits of Rs. $100$ and Rs. $80$ per piece, respectively. Production times for P and Q are $5$ hours and $3$ hours, respectively, while the total production time available is $150$ hours. For a total batch size of $40$, to maximize profit, the number of units of P to be produced is \rule{2cm}{0.15mm}}\\
\item{
Following data refers to the jobs \brak{P,Q,R,S} which have arrived at a machine for scheduling. The shortest possible average flow time is \rule{2cm}{0.15mm} days
\begin{table}[H]
    \centering
    \begin{tabular}{|c|c|}
        \hline
         Job&Processing Time$\brak{\text{days}}$  \\
         \hline
         P&9\\
         \hline
         R&22\\
         \hline
         S&12\\
         \hline
    \end{tabular}
    \label{tab:my_label}
\end{table}
}
\item{
    A block of mass $m$ rests on an inclined plane and is attached by a string to the wall as shown in the figure. The coefficient of static friction between the plane and the block is $0.25$. The string can withstand a maximum force of $20$ N. The maximum value of the mass $\brak{m}$ for which the string will not break and the block will be in static equilibrium is \rule{3cm}{0.15mm}kg. Take $\cos\theta=0.8$ and $\sin\theta=0.6$.
\begin{figure}[H]
\centering
\resizebox{3cm}{!}{%
\begin{circuitikz}
\tikzstyle{every node}=[font=\normalsize]
\draw [short] (8.5,9.75) -- (5.25,6.75);
\draw [short] (5.25,6.75) -- (8.5,6.75);
\draw [short] (8.5,9.75) -- (8.5,6.75);
\draw  (8.5,10.75) rectangle (9.75,6.75);
\draw [short] (8.5,10.25) -- (7,9);
\draw [ rotate around={46:(6.625, 8.75)}] (6.25,9.25) rectangle (7,8.25);
\node [font=\normalsize] at (6.5,8.75) {$m$};
\draw [short] (5.75,7.25) -- (6,6.75);
\node [font=\normalsize] at (6.25,7) {$\theta$};
\draw [short] (8.5,10.25) -- (9,10.75);
\draw [short] (8.5,9.75) -- (9.5,10.75);
\draw [short] (8.5,9.25) -- (9.75,10.5);
\draw [short] (8.5,8.75) -- (9.75,10);
\draw [short] (8.5,8.25) -- (9.75,9.5);
\draw [short] (8.5,7.75) -- (9.75,9);
\draw [short] (8.75,6.75) -- (9.75,7.75);
\draw [short] (8.5,7.25) -- (9.75,8.5);
\end{circuitikz}
}%

\label{fig:my_label}
\end{figure}
\begin{multicols}{4}
\begin{enumerate}
\item $10.2$
\item $10.8$
\item $9.8$
\item $9.2$
\end{enumerate}
\end{multicols}
}
\item{
A two-member truss $PQR$ is supporting a load $W$. The axial forces in members $PQ$ and $QR$ are respectively
\begin{figure}[H]
\centering
\resizebox{3cm}{!}{%
\begin{circuitikz}
\tikzstyle{every node}=[font=\normalsize]
\draw [short] (3.25,12) -- (3.25,11);
\draw [short] (3.25,8.5) -- (3.25,7.5);
\draw [short] (3.25,11.5) -- (7.25,11.5);
\draw [short] (3.25,8) -- (7.25,11.5);
\draw [short] (7.25,12) -- (7.25,9.5);
\draw [<->, >=Stealth] (3.25,11.75) -- (7.25,11.75);
\draw [->, >=Stealth] (7.25,11.5) -- (7.25,9);
\draw [short] (6.25,11.5) .. controls (6,11.25) and (6.25,10.75) .. (6.75,11);
\draw [short] (6.5,11) .. controls (6.75,10.5) and (7.25,10.5) .. (7.25,10.75);
\node [font=\normalsize] at (5.5,11) {$30\degree$};
\node [font=\normalsize] at (6.75,10.25) {60\degree};
\node [font=\normalsize] at (5.25,12) {L};
\node [font=\normalsize] at (3,11.5) {P};
\node [font=\normalsize] at (2.75,8) {R};
\node [font=\normalsize] at (7.5,11.5) {Q};
\node [font=\normalsize] at (7.5,8.75) {W};
\end{circuitikz}
}%

\label{fig:my_label}
\end{figure}
\begin{enumerate}
\item $2$W tensile and $\sqrt{3}$W compressive
\item $\sqrt{3}$W tensile and $2$W compressive
\item $\sqrt{3}$W compressive and $2$W compressive
\item $2$W compressive and $\sqrt{3}$W tensile
\end{enumerate}
\item{
A horizontal bar with a constant cross-section is subjected to loading as shown in the figure. The Young's moduli for the sections $AB$ and $BC$ are $3E$ and $E$, respectively. 
\begin{figure}[H]
\centering
\resizebox{3cm}{!}{%
\begin{circuitikz}
\tikzstyle{every node}=[font=\normalsize]
\draw [short] (5.5,11.75) -- (5.5,7.75);
\draw  (5.5,10.5) rectangle (8.5,9);
\draw  (8.5,10.5) rectangle (11.75,9);
\draw [<->, >=Stealth] (5.5,8.75) -- (8.5,8.75);
\draw [<->, >=Stealth] (8.5,8.75) -- (11.75,8.75);
\node [font=\normalsize] at (7,8.25) {L};
\node [font=\normalsize] at (10.25,8.5) {L};
\node [font=\normalsize] at (7,9.5) {3E};
\node [font=\normalsize] at (10,9.5) {E};
\node [font=\normalsize] at (6,10.75) {A};
\node [font=\normalsize] at (8.5,10.75) { B};
\node [font=\normalsize] at (11.75,10.75) {C};
\node [font=\normalsize] at (9,9.75) {P};
\node [font=\normalsize] at (12,9.75) {F};
\draw [->, >=Stealth] (9.25,9.5) -- (8.5,9.5);
\draw [->, >=Stealth] (11.75,9.5) -- (12.5,9.5);
\end{circuitikz}
}%

\label{fig:my_label}
\end{figure}
For the deflection at $C$ to be zero, the ratio $\frac{P}{F}$ is \rule{3cm}{0.15mm}
}
\item{
The figure shows cross-section of a beam subjected to bending. The area moment of inertia $\brak{\text{in }mm^4}$ of this cross-section about its base is \rule{3cm}{0.15mm}
\begin{figure}[H]
\centering
\resizebox{5cm}{!}{%
\begin{circuitikz}
\tikzstyle{every node}=[font=\normalsize]
\draw [short] (6.5,10) -- (6.5,11.5);
\draw [short] (6.5,11.5) -- (9,11.5);
\draw [short] (9,11.5) -- (9,10);
\draw [short] (6.5,7.25) -- (6.5,8.5);
\draw [short] (6.5,7.25) -- (9,7.25);
\draw [short] (9,7.25) -- (9,8.5);
\draw [short] (6.5,8.5) .. controls (7.75,8.75) and (7.75,9.75) .. (6.5,10);
\draw [short] (9,8.5) .. controls (7.75,8.75) and (7.75,9.75) .. (9,10);
\draw [<->, >=Stealth] (6.5,9.25) -- (7,9.75);
\draw [<->, >=Stealth] (9,9.25) -- (8.5,9.75);
\draw [<->, >=Stealth] (6,10) -- (6,8.5);
\draw [<->, >=Stealth] (9.5,11.5) -- (9.5,7);
\draw [<->, >=Stealth] (6.5,7) -- (9,7);
\node [font=\normalsize] at (5.75,9.25) {8};
\node [font=\normalsize] at (10,9.25) {10};
\node [font=\normalsize] at (8.5,9.25) {R4};
\node [font=\normalsize] at (7,9.25) {$R4$};
\node [font=\normalsize] at (7.75,6.75) {10};
\end{circuitikz}
}%

\label{fig:my_label}
\end{figure}
}
\end{enumerate}
\end{document}
